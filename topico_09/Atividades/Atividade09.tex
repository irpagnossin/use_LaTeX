% Este arquivo mostra como inserir uma imagem de fundo no documento através do pacote wallpaper
\documentclass[a4paper,12pt]{article}
	\usepackage[ansinew]{inputenc}
	\usepackage{ae}
	\usepackage{lipsum}	
	\usepackage[margin=3cm]{geometry}
	
	% O pacote wallpaper permite definir figuras como "papel-de-parede" do documento.
	\usepackage{wallpaper}
	
	\graphicspath{
		{Figuras/EPS/}
		{Figuras/JPG/}
		{Figuras/PDF/}
		{Figuras/PNG/}
		}	
		
\begin{document}


	% O comando \CenterWallPaper insere uma figura (segundo argumento compulsório) centralizada em todas as páginas do documento. Seu primeiro argumento define a escala, ou fator (f) de ampliação (f > 1) ou de redução (0 < f < 1).
	\CenterWallPaper{1}{Tei-Gi}
	
	% O comando \ThisCenterWallPaper tem a mesma sintaxe do \CenterWallPaper, só que ele insere a figura apenas na página do documento onde ele (comando \ThisCenterWallPaper) ocorre. Descomente a linha abaixo E COMENTE A INSTRUÇÃO ANTERIOR, \CenterWallPaper, para experimentar.
	%\ThisCenterWallPaper{1}{Tei-Gi}

	\lipsum[1-10]
	
\end{document}