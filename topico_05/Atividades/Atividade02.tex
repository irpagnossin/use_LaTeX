\documentclass[a4paper,12pt]{article}
	\usepackage[ansinew]{inputenc}
	\usepackage{ae}
	\usepackage[brazil]{babel}	
\begin{document}

	Adição: \( a + b \).
	
	Subtração: \( a - b \).
	
	Multiplicação: \( a b \) ou \( a\cdot b \) ou \( a \times b \).
	% Não use ponto-final, asterisco ou xis representar a multiplicação no LaTeX:
	% Isso é muito feio: \( a.b \) ou \(a * b\) ou \( a x b\). % Descomente esta linha para ver.
	
	Divisão: \(a/b\) ou \(\frac{a}{b}\).
	% obs.: existe também o comando TeX \over. EVITE UTILIZÁ-LO.
	
	Radiciação: \( \sqrt{a + b} \) e \(\sqrt[n]{a + b}\).

	Sobrescritos e expoentes: \( a^{b} \), \(a^{bc}\), \(a^{b^{c}}\).
	% Cuidado para não atribuir mais de um expoente ou sobrescrito a um mesmo símbolo:
	% \( a^b^c \) gerará um erro ao ser compilado (descomente esta linha e veja).
		
	Subscritos e índices: \( a_{b} \), \(a_{bc}\), \(a_{b_{c}}\).
	% Cuidado para não atribuir mais de um índice ou subscrito a um mesmo símbolo:
	% \( a_b_c \) gerará um erro ao ser compilado (descomente esta linha e veja).
	
	\bigskip % <-- Este comando do TeX aumenta um pouco a distância entre os parágrafos acima e abaixo.

	Note que \(a^bc\) é diferente de \(a^{bc}\). Analogamente, \(a_bc \ne a_{bc}\).

\end{document}