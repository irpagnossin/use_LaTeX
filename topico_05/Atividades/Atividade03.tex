\documentclass[a4paper,12pt]{article}
	\usepackage[ansinew]{inputenc}	
	\usepackage{ae}
	\usepackage[brazil]{babel}
\begin{document}

	Este é um pedaço de texto com uma expressão matemática no meio, em estilo de
	texto: \(\frac{1}{2}\). Observe que neste estilo o parágrafo prossegue 
	normalmente. Para que isto aconteça, o \LaTeX\ pode precisar reduzir a
	dimensão vertical da expressão, um comportamento ao qual você \emph{não} deve
	resistir.
		
	Este é um pedaço de texto com uma expressão matemática no meio, em estilo de
	exibição: \[\frac{1}{2}.\] Neste estilo a expressão toma uma linha inteira
	para si, ``quebrando'' o parágrafo no meio, mas não terminando-o (veja que 
	não há endentação aqui). Perceba ainda que, diferentemente do estilo de 
	texto, neste o ponto-final imediatamente após a expressão deve ser colocado
	\emph{dentro} do modo matemático (experimente colocar fora e veja o que 
	acontece).
	
	Ainda que você não deva fazer isso frequentemente, é possível suprimir a
	redução do espaço vertical ocupado por uma expressão no estilo de texto.
	Para isso, utilize o comando \verb|\displaystyle|, como aqui: \(\displaystyle
	\frac{1}{2}\). O parágrafo ainda prossegue normalmente, mas para acomodar
	esta expressão a distância entre as linhas teve de ser aumentada. Esta ação
	perturba a distribuição dos tipos no parágrafo, prejudicando a visualização.
	Às vezes é necessário, mas procure evitar fazê-lo.
	
	Analogamente, você pode requerer que a dimensão vertical de uma expressão
	matemática isolada numa linha seja reduzida, como aqui: \[\textstyle\frac{1}
	{2}.\] O comando para fazer isso é o \verb|\textstyle|.
	
	% O comando \verb impede a interpretação de seu argumento pelo LaTeX, 
	% permitindo-lhe escrever caracteres especiais, como a barra invertida,
	% sem utilizar comandos do LaTeX (\textbackslash, por exemplo). 
	% Diferentemente de qualquer outro comando, o argumento de \verb não é
	% delimitado por chaves, mas por qualquer caráter não-alfabético: você
	% só precisa utilizá-lo no início e no fim do argumento. Mas atenção: o
	% comando \verb NÃO pode ser utilizado como argumento de outros comandos.	
	
\end{document}