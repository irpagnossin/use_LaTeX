\documentclass[a4paper,10pt]{article}
	\usepackage[ansinew]{inputenc}
	\usepackage{ae}
	\usepackage[brazil]{babel}
	\usepackage{xcolor} % <-- O pacote xcolor é uma evolução do color.
	%\usepackage{amsmath}
	
	% Descomente as três linhas abaixo para alterar a fonte do texto e evidenciar a diferença entre as fontes do modo parágrafo e do modo matemático.
%	\renewcommand\familydefault{fau}
%	\renewcommand\seriesdefault{m}
%	\renewcommand\shapedefault{n}
	
	% Eu acabei de inventar estes comandos. Por que eu fiz isso? Discuta com seus colegas na atividade conjunta deste tópico. Mas atenção: não se preocupe em entender as instruções abaixo... por enquanto!
	\newcommand{\pacote}[1]{{\color{orange!50!black}\normalfont#1}}
	\newcommand{\comando}[1]{{\textbackslash\color{blue!70}\normalfont#1}}
		
\begin{document}

	\section*{Textos no modo matemático}

	A forma tradicional de inserir textos no modo matemático é através do
	comando \comando{mbox} (``make box''):
	
	\begin{center}
		texto \( \cos\left( \mbox{ângulo qualquer} \right) \) texto.
	\end{center}	
	Neste caso, a fonte utilizada para escrever o argumento do \comando{mbox}
	é aquela em voga no modo parágrafo (experimente alterá-la).
	
	% Na verdade, o comando \mbox causa uma transição para o chamado
	% modo LaTeX LR (left-right), ou modo horizontal restrito (TeX).
	
	Você também pode utilizar qualquer um dos comandos ordinários de seleção
	de fontes vistos anteriormente caso queira especificar, explicitamente,
	os atributos da fonte. Por exemplo, se você quiser escrever dentro do
	modo matemático com a mesma fonte do modo texto, mas com a forma inclinada
	(slanted), utilize \comando{textsl}:
	\begin{center}
		texto \( \cos\left( \textsl{ângulo qualquer} \right) \) texto.
	\end{center}	
	Novamente, experimente alterar a família ou a série da fonte no \emph{modo
	parágrafo}.
	
	A desvantagem desses comandos é que, se você utilizá-los em subscritos ou
	sobrescritos, terá de ajustar o tamanho da fonte manualmente, o que não é
	logicamente recomendado (além de ser trabalhoso):
	\begin{center}
		texto \( \cos\left( \phi_{\textsl{inicial}} \right) \) texto.
	\end{center}	
	
	Para resolver isso, simplesmente carregue o pacote \pacote{amsmath} no preâmbulo:
	ele redefine esses comandos levando essa questão em consideração. Faça isso agora
	e recompile este arquivo.
	
	O pacote \pacote{amsmath} também define o comando \comando{text}, que deve ser utilizado
	no lugar do \comando{mbox} pelo mesmo motivo. Este é o comando que você mais utilizará
	para inserir textos no modo matemático.

	%-----------------------------------
	\section*{Fontes do modo matemático}
	
	Diferentemente de inserir textos no modo matemático, você pode querer modificar os
	atributos das fontes \emph{do} modo matemático. Isto é feito através dos comandos
	\comando{mathrm}, \comando{mathsf}, \comando{mathtt}, \comando{mathbf}, \comando{mathit}
	e \comando{mathcal}. Ademais, o comando \comando{mathnormal} restaura a fonte padrão do modo matemático.
	
	Estes comandos não promovem a transição para algum outro modo, mas apenas alteram
	os atributos da fonte no modo matemático. E por permanecer neste modo, o argumento
	desses comandos estão sujeitos às regras do modo matemático. Isto é, os espaços
	são ignorados e você não pode utilizar acentos:
	\[\mathrm{ABCD}\quad
	  \mathsf{ABCD}\quad
	  \mathtt{ABCD}\quad
	  \mathbf{ABCD}\quad
	  \mathit{ABCD}\quad
	  \mathcal{A  B  C  D}\] % <-- Note que os espaços são ignorados: característica do modo matemático.
	
	Estes comandos alteram automaticamente o tamanho das fontes caso estejam
	como sub ou sobrescritos.
\end{document}