\documentclass[a4paper,12pt]{article}
	\usepackage[ansinew]{inputenc}
	\usepackage{ae}
	\usepackage[brazil]{babel}
\begin{document}

	% ----------------------------
	% 1ª regra do modo matemático: espaços (e quebras de linha) são ignorados
	% ----------------------------
	% No modo matemático os espaços entre os símbolos são gerenciados pelo LaTeX
	% com base na função deles. Por exemplo, ele sabe que o símbolo + representa 
	% uma operação binária, e que "a" e "b" são variáveis. Por conta disso, todo
	% o espaço digitado por você no arquivo de instruções é ignorado.
	\(a +
        b =
            c\) e     \( a + b = c \)    são equivalentes.
  
  
  % ----------------------------
  % 2ª regra do modo matemático: linhas em branco são proibidas.
  % ----------------------------
  % Experimente inserir \par ou uma linha em branco acima


	% ----------------------------
	% 3ª regra do modo matemático: acentos são proibidos.
	% ----------------------------
	%\( Se você descomentar esta linha, obterá avisos do compilador (leia essas mensagens). \)
		
	

	% obs.: JAMAIS USE O MODO MATEMÁTICO PARA ESCREVER EM ITÁLICO!

\end{document}