\documentclass{article}
	\usepackage[ansinew]{inputenc}
	\usepackage{ae}
	\usepackage[brazil]{babel}
	\usepackage{amsmath}
	
	\DeclareMathOperator{\sen}{sen} % <-- Só no preâmbulo. Requer o pacote amsmath.
	
\begin{document}
	
	\[ \sin\left( A + B \right) = \sin A \cos B + \sin B \cos A \]
	% Observe que o argumento da função seno (cosseno) não tem nada a ver com
	% o comando \sin (\cos). Isto é, ele não faz parte do comando: você pode
	% escrever \(\sin\) sem problema algum, embora isto geralmente não ocorra.

	
	\[\sen^2 \theta + \cos^2 \theta = 1\] % <-- "sen" ao invés de "sin"
	
\end{document}