\documentclass[a4paper,12pt]{letter}

	\usepackage[ansinew]{inputenc}
	\usepackage{bookman} % <-- No lugar do pacote "ae"
	\usepackage[brazil]{babel}
	\usepackage{usandoLaTeX}	

	\name{Sr. Leão \TeX}
	
	\address{Instituto de Física da USP/SP \\
           Rua do Matão, Travessa R, nº 187 \\
           Edifício do Milênio --- Cidade Universitária/USP \\
           São Paulo --- SP}
           
	\signature{A mascote do \TeX, Sr. Leão}
	\date{São Paulo, \today}

	\makelabels
\begin{document}
	\begin{letter}{À sala de aulas do curso ``Usando \LaTeX, pensando em \TeX'' \\
                 A/C do \foreign{\TeX starter} que lê esta carta}

	\opening{Ilmo(a). Sr(a).,}

	É com imenso prazer que venho, através desta carta, manifestar meu apreço por
	seu esforço em acompanhar este curso de \LaTeX. Ao longo do último mês você 
	aprendeu a utilizar corretamente o \LaTeX, algo que --- infelizmente --- não
	é muito comum: conceitos como caixas, comprimentos elásticos e organização
	lógica costumam ser deixados de lado, o que é uma pena. Mas você teve ânimo e
	força de vontade para prosseguir nesta jornada que, tenho certeza, tem sido
	muito interessante (caso contrário você já teria desistido, não é?).

	Por isso eu o(a) parabenizo, com a certeza de que sua recompensa virá não
	apenas com os benefícios desta ferramenta, mas também com uma forma diferente
	de pensar e de escrever.	

	\closing{Atenciosamente,}

	\cc{Dr. Ivan Ramos Pagnossin, o prof.}
	\ps{\textbf{obs.:} o curso ainda não acabou. Por isso, mãos à obra!}
	\end{letter}
	
\end{document}