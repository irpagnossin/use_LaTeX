\documentclass[a4paper,12pt]{article}
%\documentclass[a4paper,12pt]{scrartcl} % <-- Faz parte do "bundle" Koma-Script

	\usepackage[ansinew]{inputenc}
	\usepackage{ae}
	\usepackage[brazil]{babel}		
	\usepackage{amstext}
	\usepackage{indentfirst}
	
	%----------------------------------------------------------------------------
	% O comando \input abaixo insere o arquivo defs.tex nesta posição, como se 
	% todo o conteúdo daquele arquivo tivesse sido digitado aqui. Particularmente
	% neste caso, o arquivo defs.tex contém algumas definições de comandos 
	% utilizadas aqui.
	%----------------------------------------------------------------------------
	\input{usandoLaTex}
	%\usepackage{usandoLaTeX}
			
	%----------------------------------------------------------------------------
	% O comando \title define o título do trabalho, mas NÃO o imprime (isto é
	% feito pelo comando \maketitle, mais a frente).
	%----------------------------------------------------------------------------
	\title{O título deste trabalho}
	
	%----------------------------------------------------------------------------
	% O comando \author define o(s) autor(es) do trabalho. Cada autor deve ser
	% separado pela instrução \and. O comando \thanks insere uma nota de rodapé,
	% gealmente utilizada (neste contexto) para informar e-mails.
	%----------------------------------------------------------------------------
	\author{
		Fulano de tal\thanks{fulano@latex.br}\\
		Instituto dos fulaninhos
		\and
		Sicrano de tal\thanks{sicrano@tex.br}\\
		Universidade dos sicraninhos}
		
	%----------------------------------------------------------------------------
	% O comando \date define a data do trabalho. Você pode escrevê-la no formato
	% que quiser ou omití-lo. Neste caso, o LaTeX inserirá automaticamente a data
	% atual no formato padrão (não se esqueça de utilizar o pacote babel ou ela
	% será apresentada no formato americano).
	%----------------------------------------------------------------------------
	\date{São Paulo, \today}
	
\begin{document}

	\pagenumbering{Roman}

	\maketitle
		
	\begin{abstract}
		Para escrever um resumo (ou abstract) do trabalho, utilizamos o ambiente
		\env{abstract}. A formatação é totalmente gerenciada pelo \LaTeX,	restando
		ao autor apenas preocupar-se com o conteúdo.
		
		Note também a ação do pacote \pkg{babel}: sem ele, ao invés do título
		``resumo,''	teríamos ``abstract.'' Experimente remover o comando de 
		carregamento do pacote (\cs{usepackage}).\footnote{\emph{Atenção:} remover
		o pacote \pkg{babel} após uma primeira compilação com ele pode gerar erro
		na compilação seguinte. Se isto ocorrer, simplesmente recompile (pergunte
		ao professor o porquê disto).}
	\end{abstract}
	
	\pagenumbering{roman}
	
	\section{Posicionamento padrão de expressões matemáticas}
	
	As expressões matemáticas em estilo de exibição geralmente são posicionadas
	no centro de sua própria linha, com a numeração, se houver, junto à margem
	direita:
	%\setlength{\mathindent}{2\parindent} % <-- Para descomentar esta linha, você
	                                     % deve utiliza a opção fleqn da classe.
	\begin{equation}\label{eq:QHE}
		R_H = \frac{R_K}{N}\text{.}
	\end{equation}
	
	Mas você pode alterar este posicionamento: as classes-padrão do \LaTeX\
	disponibilizam as opções \opt{leqno} (\foreign{left equation number}) e
	\opt{fleqn} (\foreign{flushed left equation}). A primeira faz com que
	os números das expressões sejam	colocados na margem	esquerda da linha. Já a
	opção \opt{fleqn} instrui o	\LaTeX\ a posicionar a expressão à esquerda, a
	uma distância parametrizada	por \cs{mathindent}. Mas atenção: \cs{mathindent}
	só é definido se a opção \opt{fleqn} tiver sido utilizada. Isto significa que
	se você tentar alterá-lo sem indicar explicitamente a opção \opt{fleqn},
	obterá um erro de compilação (\foreign{undefined control sequence}).
		
	\begin{thebibliography}{0}
		\bibitem{kopka:1998} H. Kopka e P. W. Daly, \textsl{A guide to \LaTeX},
		Addison-Wesley, 1998.
	\end{thebibliography}
	
\end{document}