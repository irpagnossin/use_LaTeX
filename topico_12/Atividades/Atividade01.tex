\documentclass[a4paper,12pt]{article}
	\usepackage[ansinew]{inputenc}
	\usepackage{ae}
	\usepackage[brazil]{babel}
	\usepackage{hyperref}
	\usepackage{usandoLaTeX}
	
\begin{document}

	A forma mais simples de se montar uma lista de referências bibliográficas
no \LaTeX\ é à mão, através do ambiente \env{thebibliography} e dos
comandos \cs{bibitem} (\foreign{bibliographic item}) e \cs{cite}.

	O ambiente \env{thebibliography} define uma lista cujos itens são
nomeados e, por isso, podem ser referenciados ao longo do texto:
		
	\begin{thebibliography}{-}
	
	  \bibitem{Kopka:1999}          % Rótulo
	    H. Kopka e P. W. Daly,      % Autor(es)
	    \textsl{A guide to \LaTeX}, % Título
	    Addison-Wesley              % Editora
	    (1999).                     % Ano
	    
	  \bibitem{Gratzer:1996}        % Rótulo
	   	G. Grätzer,                 % Autor(es)
	   	\textbf{Math into \LaTeX},  % Título
	   	\textit{Birkhäuser}         % Editora
	   	(1996).                     % Ano
	   	
		\bibitem{e-mail}
			Contato: e-m@il.com
			
	\end{thebibliography}

	Cada item da lista \env{thebibliography} é iniciado pelo comando
\cs{bibitem}, que recebe como argumento compulsório o rótulo, ou nome,
que se lhe quer atribuir. Este comando é análogo ao \cs{label}, que
utilizamos para nomear equações, figuras, tabelas, seções, etc.

	Com a lista definida, podemos citar qualquer um de seus itens utilizando
o comando \cs{cite}, cujo argumento obrigatório é o rótulo do item
que ser quer referenciar. Por exemplo, o comando \cs[Gratzer:1996]{cite} gera
uma referência para o item \cite{Gratzer:1996} da lista, isto é, o livro de
G.~Grätzer. É simples assim. E como no caso de equações, figuras, etc, qualquer
alteração na lista de referências é automaticamente tratada pelo \LaTeX\ (você
deve apenas lembrar-se de compilar duas vezes).
	
	% Observe que na primeira compilação e após qualquer alteração nos
	% rótulos da lista, você obtém o aviso
	%
	% LaTeX Warning: Label(s) may have changed. Rerun to get cross-references right.
	
	Note que o conteúdo do (bib)item é totalmente arbitrário, como ilustra o
item \cite{e-mail}. Realmente: você pode ainda inserir figuras, tabelas
etc, embora isto não seja usual. Além disso, o título ``Referências'' é
inserido automaticamente.

	% Você obterá "References" se tiver esquecido de carregar o pacote babel
	% (opção "brazil").

	Finalmente, observe que o ambiente \env{thebibliography} requer um
parâmetro obrigatório (um ``\texttt{-}\rlap{,}'' no caso). Trata-se de uma
\emph{estimativa} do espaço horizontal ocupado por cada rótulo que precede o
item \emph{no documento de saída}. Dito de outra forma, é uma estimativa do
espaço ocupado pelo ``1'' em ``[1]'' etc (uma regra prática é usar 9 para
até nove referências, 99 para até noventa e nove, e assim sucessivamente).

	\vfill % Análogo a \hfill, só que na vertical: "vertical fill"
	
\noindent\textbf{dica:} se você estiver utilizando o pacote \texttt{hyperref}
(veja o preâmbulo), o comando \cs{cite} gerará também um \foreign{hiperlink}
para o item da lista. Experimente: insira o pacote e recompile!

\end{document}