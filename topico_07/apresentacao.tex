%-------------------------------------------------------------------------------
% Autores: I. R. Pagnossin e Centro de Ensino e Pesquisa Aplicada.
%
% Este material é parte integrante do curso "Usando LaTeX; pensando em TeX" e é
% distribuido pelos autores segundo a licença Creative Commons 2.5 Brasil
% (atribuição/não-comercial/redistribuição segundo a mesma licença).
%
% This material is part of the course "Usando LaTeX; pensando em TeX".
% It is distributed according to the license Creative Commons 2.5 Brazil
% (attribution/non-comercial use/share alike the same license).
%-------------------------------------------------------------------------------
\newif\ifhandout
%\handouttrue  % Descomente se for para gerar a versão para IMPRESSÃO.
\handoutfalse % Descomente se for para gerar a versão para APRESENTAÇÃO

%-------------------------------------------------------------------------------
\ifhandout
  \documentclass[handout,10pt]{beamer}
  \mode<handout>
\else
	\documentclass[10pt,hyperref={pdfpagelabels=false}]{beamer}
	\mode<presentation>
\fi

	\usepackage[utf8]{inputenc}
	\usepackage{ae}	
	\usepackage[brazil]{babel}	
	\usepackage{graphicx}
	\usepackage{listings}
	\usepackage{tikz}
	\usepackage{amsmath}
	\usepackage[squaren]{SIunits}
	\usepackage{booktabs}
	\usepackage{fancybox}
	\usepackage{array}
	\usepackage{colortbl}
		
	\usepackage{bookman}
	
	\ifhandout
		\usepackage{pgfpages}
		\pgfpagesuselayout{2 on 1}[a4paper,border shrink=5mm]
	\fi

	% Bibliotecas TikZ e PGF necessárias
	\usetikzlibrary{shapes.symbols}
	\usetikzlibrary{calc}
	\usepgflibrary{shapes.misc}
	
	% Configurações pessoais
	% Configurações personalizadas do código LaTeX.	
\lstnewenvironment{LaTeXcode}{
	\setlength{\abovecaptionskip}{0pt}	
	\lstset{language=[LaTeX]TeX}
	\lstset{%
		basicstyle=\footnotesize\ttfamily,  % Global
		keywordstyle=\color{blue}\bfseries, % Comandos
		identifierstyle=,                   % Texto
		stringstyle=,                       % Strings 
		commentstyle=\color{gray},          % Comentários
		showstringspaces=false,             % Espaços
		rulecolor=\color{gray},             % Linha da caixa
	}
	\lstset{emph={setlength,includegraphics,psfrag,subfigure},emphstyle={\color{blue}\bfseries}}
}% Abrindo o ambiente.
{}% Fechando o ambiente.
	
\newcommand{\digite}[1]{{\fontfamily{cmss}\fontseries{bx}\selectfont#1}}	
\newcommand{\cs}[1]{{\normalfont\textbackslash\color{blue!50!black}#1}}
\newcommand{\pkg}[1]{{\normalfont\sffamily\color{orange}#1}}
\newcommand{\env}[1]{{\normalfont\sffamily\color{green!50!black}#1}}
\let\comando=\cs
\let\package=\pkg
\let\ambiente=\env
\newcommand{\foreign}[1]{{\textsl{#1}}}


	\newcounter{exercicio}	
	\newenvironment{exercicio}{%
		\refstepcounter{exercicio}%
		\penalty-200
		\noindent\colorbox{blue!60!black}{\makebox[\columnwidth-\fboxsep*2][c]{\textbf{\color{white}Exercício~\theexercicio}}}\smallskip
	}{\par\medskip}
		

\newcommand{\bibtex}{\textsc{Bib}\TeX}

\newenvironment<>{atividade}[1]{%
\begin{actionenv}#2%
\begin{exampleblock}{{Atividade #1}}%
}
{%
\end{exampleblock}%
\end{actionenv}%
}


	% Path das figuras, relativo a esta pasta.
	\graphicspath{{../arquivos_comuns/figuras/}{./figuras/}}

	% Modelo da apresentação	
	\usetheme{Frankfurt}
	\usefonttheme{serif,structurebold}
	\setbeamercovered{transparent}		
		
	% Metadados do arquivo PDF.
	\hypersetup{
		pdftitle={Comandos personalizados, listas e contadores},
		pdfauthor={Dr. Ivan R. Pagnossin},
		pdfsubject={LaTeX},
		pdfkeywords={TeX,LaTeX}
	}

	% Título, autores e instituição.
	\title{Comandos personalizados,}
	\subtitle{listas e contadores}
	\author{\textbf{Prof.:} Ivan R. Pagnossin \and \textbf{Tutora:} Juliana Giordano}
	\institute{%
		Coordenadoria de Tecnologia da Informação\\
		Centro de Ensino e Pesquisa Aplicada}
	\logo{\includegraphics[width=0.25\textwidth]{LogotipoCursoLaTeX_v3_pequeno}}
	\date{}
	
\begin{document}


%-------------------------------------------------------------------
\begin{frame}[c,label=titulo]
	\centering	
	
	\includegraphics[width=0.8\textwidth]{LogotipoCursoLaTeX_v2}

	\titlepage
\end{frame}
\end{document}
%-------------------------------------------------------------------
\logo{} % <-- O logotipo não aparecerá mais a partir daqui.
\setbeamertemplate{background canvas}{%
		\includegraphics[width=\paperwidth,height=\paperheight,keepaspectratio=false]{leao-pensador-wattermark.png}
}
%-------------------------------------------------------------------
\begin{frame}[fragile]
	\frametitle{Listas}
	\framesubtitle{\foreign{itemize}, \foreign{enumerate} e \foreign{description}}
	
	\begin{columns}
		\scriptsize
		
		\column{0.3\textwidth}
		\begin{block}{\foreign{itemize}}
			\begin{itemize}	
				\item Maçã é vermelha
				\item Uva é roxa
				\item Banana é amarela
				\item Laranja é\dots\ laranja!
			\end{itemize}
		\end{block}
		
		\column{0.3\textwidth}		
		\begin{block}{\foreign{enumerate}}
			\begin{enumerate}	
				\item Maçã é vermelha
				\item Uva é roxa
				\item Banana é amarela
				\item Laranja é\dots\ laranja!
			\end{enumerate}
		\end{block}
		
		\column{0.35\textwidth}
		\begin{block}{\foreign{description}}			
			\begin{description}
				\item[Maçã] é vermelha
				\item[Uva] é roxa
				\item[Banana] é amarela
				\item[Laranja] é\dots\ laranja!
			\end{description}
		\end{block}
	\end{columns}
	
	\vfill
	
	\begin{block}{}
		\centering
		\verb|\begin{|\textit{lista}\verb|}|%
		\quad\dots\quad
		\verb|\item[|\textit{símbolo}\verb|]|%
		\quad\dots\quad
		\verb|\end{|\textit{lista}\verb|}|
	\end{block}
	
	\vfill
	
	\begin{atividade}{1}
		\begin{LaTeXcode}
		\begin{itemize}	
		  \item Maçã é vermelha
		  \item[$\star$] Uva é roxa
		  \item Banana é amarela
		  \item Laranja é\dots\ laranja!
		\end{itemize}
		\end{LaTeXcode}
	\end{atividade}
	
\end{frame}
%-------------------------------------------------------------------
\begin{frame}[fragile]
	\frametitle{Comandos personalizados}
	
	\begin{block}{}
		\centering
		\verb|\newcommand{\|\textit{nome do comando}\verb|}[|\textit{n\textordmasculine\ de arg.}\verb|]{|\textit{definição}\verb|}|
	\end{block}
	
	\begin{atividade}{2}
		A instrução abaixo define o comando \cs{avec}:
		\begin{LaTeXcode}
			\newcommand{\xvec}{ x_1, \ldots, x_n }
		\end{LaTeXcode}
		
		Assim, escrever \verb|$\avec$| é o mesmo que escrever \verb|$x_1, \ldots, x_n$|
	\end{atividade}
	
\end{frame}
%-------------------------------------------------------------------
{
	\logo{\includegraphics[width=0.25\textwidth]{LogotipoCursoLaTeX_v3_pequeno}}
	\setbeamertemplate{background canvas}{}
	\againframe{titulo} % Reapresenta a página inicial.
}
%-------------------------------------------------------------------
%-------------------------------------------------------------------
%-------------------------------------------------------------------
\section{Apêndice}
\subsection{Respostas}
\begin{frame}[fragile,label=respostas]
	\frametitle{Respostas}
	\scriptsize

\end{frame}
\end{document}