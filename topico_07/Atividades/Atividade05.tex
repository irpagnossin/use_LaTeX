\documentclass[a4paper,12pt]{article}
	\usepackage[ansinew]{inputenc}
	\usepackage{ae}
	\usepackage[brazil]{babel}
	\usepackage{xcolor}
	\usepackage{lipsum} % Lorem ipsum
\begin{document}

	% O ambiente list representa uma lista generalizada que você pode configurar.

	\lipsum[1] % Gera o texto anterior à lista.

	% Olhe só: trata-se de um ambiente que recebe DOIS argumentos! A sintaxe
	% é \begin{list}{arg. 1}{arg. 2}... você enxerga os argumento abaixo?
	%
	\begin{list}
		% 1º argumento
		%-------------
		{\( \to \)}		
		% 2º argumento
		%-------------
		{
			\setlength{\itemindent}{\parindent}
			\setlength{\itemsep}{0pt}
			\setlength{\rightmargin}{\leftmargin}
		}
	
	
	
	
		\item \lipsum[2] % Item 1.
		\item \lipsum[3] % Item 2.
	\end{list}

	
	\lipsum[4] % Gera o texto posterior à lista.

\end{document}