\documentclass[a4paper,12pt]{article}
	\usepackage[ansinew]{inputenc}
	\usepackage{ae}
	\usepackage[brazil]{babel}	
	\usepackage{xcolor} % <-- O pacote "xcolor" é uma evolução do "color".
\begin{document}

	% Os símbolos da lista "itemize" são definidos pelos comandos \labelitemi,
	% \labelitemii, \labelitemiii e \labelitemiv, cujas definições são apenas o
	% símbolo a ser introduzido por \item.
	
	% Experimente redefiní-los, como abaixo, e veja o resultado.

	\renewcommand{\labelitemi}  {\( \Rightarrow \)}
	\renewcommand{\labelitemii} {\( \to         \)}
	\renewcommand{\labelitemiii}{\( +           \)}
	\renewcommand{\labelitemiv} {\textcolor{red}{:-)}}


	\begin{itemize}
		\item Item 1 do nível 1

		\begin{itemize}
			\item Item 1 do nível 2

			\begin{itemize}
				\item Item 1 do nível 3

				\begin{itemize}
					\item Item 1 do nível 4
				\end{itemize}
			\end{itemize}
		\end{itemize}

		\item Item 2 do nível 1
		\item etc
	\end{itemize}

\end{document}
