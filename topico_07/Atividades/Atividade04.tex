\documentclass[a4paper,12pt]{article}
	\usepackage[ansinew]{inputenc}
	\usepackage{ae}
	\usepackage[brazil]{babel}
	\usepackage{xcolor}
		
	% Definir o comando \contador para formatar cada ocorrência de um contador é
	% uma maneira logicamente correta de organizar o documento, mas utilizar os
	% diversos \verb abaixo não é... mas dá muito trabalho definir um comando
	% suficientemente abrangente para representar um comando genérico. E como o
	% documento é curto, escolhi não fazê-lo.
	\newcommand{\contador}[1]{\textit{#1}}
\begin{document}

	Na lista abaixo, a banana aparece no item de número \ref{item:banana}. Este é
	um exemplo de referência-cruzada, como aquela que vimos no caso de equações.

	\begin{enumerate}
		\item Maçã é vermelha
		\item \label{item:banana} Banana é amarela %\stepcounter{enumi}
		\item Uva é roxa
		
		\item Laranja é\dots\ laranja!
	\end{enumerate}

	O mecanismo de referências-cruzadas baseia-se nos contadores: comandos que
	representam números inteiros. Eles numeram ítens, capítulos, seções, figuras,
	equações e tudo o mais que a sua imaginação quiser.

	Por exemplo, os números na lista anterior são armazenados pelo contador
	\contador{enumi}. Para cada novo \verb|\item|, o \LaTeX\ adiciona 1 (um) a
	este contador e imprime-o antes do texto (inicialmente \(\contador{enumi} =
	0\)).
	
	Com efeito, o comando \verb|\label| simplesmente anota no arquivo auxiliar
	(aquele com extensão \texttt{aux}) o valor do contador \contador{enumi} 
	naquele ponto do documento.
	
	As classes-padrão do \LaTeX\ já definem uma série de contadores, cada qual
	para uma estrutura do documento:
		\contador{part},
		\contador{chapter},
		\contador{section},
		\contador{subsection},
		\contador{subsubsection},
		\contador{paragraph},
		\contador{subparagraph},
		\contador{page},
		\contador{equation},
		\contador{figure},
		\contador{table},
		\contador{footnote},
		\contador{mpfootnote},
		\contador{enumi},
		\contador{enumii},
		\contador{enumiii} e
		\contador{enumiv}.
				
	Você pode definir o seu próprio contador para, digamos, numerar uma lista 
	de exercícios. Chamemos este contador de \contador{exercicio} (uma palavra
	qualquer sem acentos, espaços ou outros caracteres especiais) e utilizemos
	o comando \verb|\newcounter{|\contador{exercicio}\verb|}| para definí-lo.
	
	\newcounter{exercicio}
	
	O valor inicial do contador é zero. Para alterá-lo, você pode utilizar os
	seguintes comandos:
	\begin{itemize}
		\item
		\verb|\setcounter{|\contador{exercicio}\verb|}{|\textit{número}\verb|}|:
		atribui o \textit{número} ao contador.
		
		\item
		\verb|\addtocounter{|\textit{contador}\verb|}{|\textit{número}\verb|}|:
		adiciona o \textit{número} ao contador.
					
		\item
		\verb|\stepcounter{|\textit{contador}\verb|}|:
		adiciona 1 (um) ao contador.
					
		\item	\verb|\refstepcounter{|\textit{contador}\verb|}|:		
		idem ao anterior, mas atualiza o contador utilizado para as
		referências-cruzadas (que podem ser acessados pelo comando \verb|\ref|).
		
	\end{itemize}	
		
	Para \emph{imprimir} o valor do contador, utilize os comandos 
		\verb|\arabic{|\contador{exercicio}\verb|}|,
		\verb|\roman{|\contador{exercicio}\verb|}|,
		\verb|\Roman{|\contador{exercicio}\verb|}|,
		\verb|\alph{|\contador{exercicio}\verb|}|,
		\verb|\Alph{|\contador{exercicio}\verb|}| ou ainda
		\verb|\fnsymbol{|\contador{exercicio}\verb|}|:
	
	% Experimente alterar o valor atribuido ao contador.
	\setcounter{exercicio}{2}
	
	\begin{center}
		\arabic{exercicio},
		\roman{exercicio},
		\Roman{exercicio},
		\alph{exercicio},
		\Alph{exercicio} ou
		\fnsymbol{exercicio}.
	\end{center}
	
	Além deles, para cada \contador{contador}, o \LaTeX\ cria automaticamente o
	comando \verb|\the|\contador{contador}, que costuma ser utilizado para 
	definir um formato conveninente de exprimir o \contador{contador}.
	
	Por exemplo, pode ser conveniente que nossos exercícios sejam numerados por
	seção: eg, ``exercício 3.7'' representaria o sétimo exercício da seção 3.
	Para fazermos isso, basta redefinirmos \verb|\theexercicio|.
	
	% Por que usamos \renewcommand ao invés de \newcommand?
	\renewcommand{\theexercicio}{\arabic{section}.\arabic{exercicio}}
	
	Aqui, como não criamos nenhuma seção, o contador \contador{section} é igual a
	zero. Já \contador{exercicio} é igual a 2, pois foi este o último valor que
	lhe atribuímos, algumas linhas acima. Logo, o comando \verb|\theexercicio|
	resulta em \theexercicio.
	
\end{document}
