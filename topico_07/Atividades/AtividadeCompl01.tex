\documentclass[a4paper,12pt]{article}
	\usepackage[ansinew]{inputenc}
	\usepackage{ae}
	\usepackage[brazil]{babel}
	\usepackage{paralist}
	\usepackage{xcolor} % O pacote xcolor substitui o color, mais antigo.
	
	% O argumento "blue!60!white", definido pelo pacote xcolor, indica a cor do
	% texto. Neste caso, esta cor é constituída de uma mistura de 60% de azul e
	% 40% de branco.
	\newcommand{\pacote}[1]{\textcolor{blue!60!white}{#1}}
	\newcommand{\ambiente}[1]{\textsl{#1}}
	
\begin{document}

	O espaço entre os itens das listas-padrão do \LaTeX\ é consideravelmente
	maior que a distância entre as linhas, o que as tornam pouco atrativas no
	caso de ítens curtos:

	\begin{itemize}
		\item Maçã é vermelha
		\item Uva é roxa
		\item Banana é amarela
		\item Laranja é\dots\ laranja!
	\end{itemize}
	
	Mas você pode definir sua própria lista, com os comprimentos que mais achar
	adequados. Alternativamente, você pode simplesmente importar o pacote \pacote
	{paralist} para o seu documento e utilizar o ambiente \ambiente{compactitem}
	no lugar do \ambiente{itemize}, \ambiente{compactenum} no de
	\ambiente{enumerate} e, finalmente, \ambiente{compactdesc} no lugar de
	\ambiente{description}:

	\begin{compactitem}
		\item Maçã é vermelha
		\item Uva é roxa
		\item Banana é amarela
		\item Laranja é\dots\ laranja!
	\end{compactitem}
	
	% EXERCÍCIO: crie as listas "compactenum" e "compactdesc".

\end{document}
