\documentclass[a4paper,12pt]{article}
	\usepackage[ansinew]{inputenc}
	\usepackage{ae}
	\usepackage[brazil]{babel}
	\usepackage{xcolor}
	\usepackage{lipsum} % Lorem ipsum
\begin{document}

	\newcounter{meucontador}
	
	% O comando \newenvironment cria um ambiente novo. O primeiro argumento
	% é o nome deste ambiente, o segundo contém as instruções que devem ser
	% executadas no \begin, enquanto o terceiro contém as instruções que devem
	% ser executadas no \end.
	\newenvironment
		% ------------------
		% Primeiro argumento		
		% ------------------
		{minhalista}
		% -----------------
		% Segundo argumento
		% -----------------
		{\begin{list}
			{(\roman{meucontador})}
			{\usecounter{meucontador}
			 \setlength{\itemindent}{\parindent}
			 \setlength{\itemsep}{0pt}
			 \setlength{\rightmargin}{\leftmargin}
			}
		}
		% ------------------
		% Terceiro argumento
		% ------------------
		{\end{list}}
	
	
	
	
	% Agora que a "minhalista" está definida, podemos utilizá-la!
	\lipsum[1]
	
	\begin{minhalista}
		\item \label{paragrafo 2}
					\lipsum[2] % Item 1.
		
		\item \lipsum[3] % Item 2.
					
		\item \lipsum[4] % Item 3.
		
		\item Maçã é vermelha		
		\item Uva é roxa
		\item Banana é amarela
		\item Laranja é\dots\ laranja!		
	\end{minhalista}
	
	
	O segundo parágrafo do texto \textit{Lorem ipsum} está no item
	\ref{paragrafo 2} da lista acima.

	% obs.: analogamente ao \newcommand, existe também o \renewenvironment,
	% cuja sintaxe é exatamente a mesma do \newenvironment e cuja utilização
	% você pode deduzir.

\end{document}