\documentclass[a4paper,12pt]{article}
	\usepackage[ansinew]{inputenc}
	\usepackage{ae}
	%\usepackage[brazil]{babel} % <-- Traduz "References" para "Referências"
	\usepackage{xcolor}	
\begin{document}

	% Este é um exemplo de lista de referências bibliográficas. Ela é um pouco
	% diferente das anteriores, mas a idéia é a mesma. Sua vantagem é que o
	% comando \bibitem, análogo ao \item, recebe um argumento obrigatório que é
	% justamente o rótulo daquele item (ou daquela referência, se você preferir).
	% Isto evita que você tenha de ficar digitando \label para cada item.
	
	Citando Kopka: \cite{Kopka:1999}.
	
	\begin{thebibliography}{+}
	
	  \bibitem{Kopka:1999}          % Rótulo
	    H. Kopka e P. W. Daly,      % Autor(es)
	    \textsl{A guide to \LaTeX}, % Título
	    Addison-Wesley              % Editora
	    (1999).                     % Ano
	    
	  \bibitem{Gratzer:1996}        % Rótulo
	   	G. Grätzer,                 % Autor(es)
	   	\textbf{Math into \LaTeX},  % Título
	   	\textit{Birkhüaser}         % Editora
	   	(1996).                     % Ano
	   	
		\bibitem{e-mail}
			Talvez meu e-mail e telefone para contato.
			
	\end{thebibliography}	

	% Perceba que o ambiente thebibliography recebe um argumento ("+", no caso).
	% Trata-se apenas de um ou mais caracteres cuja dimensão aproxima-se daquela
	% utilizada pelos símbolos que precedem os ítens. Ele serve para que o LaTeX
	% possa reservar a quantidade certa de espaço para esses símbolos, visando
	% manter os ítens alinhados. É comum você ver "9" ou "99" no lugar de "+" e
	% talvez seja o mais indicao. Isto é, use "9" se você tem até 9 referências,
	% "99" se você tem até 99 e assim por diante.
	
	% Você percebeu que, embora o ambiente thebibliography ajude a construir uma
	% lista de referências bibliográficas, o leiaute de cada item é organizado
	% por você, o autor? Isto vai de encontro com os princípios de organização
	% lógica do LaTeX e há uma maneira melhor de gerenciar referências: através
	% de um programa chamado BibTeX. Mas isto uma história para outra ocasião.
	
	
	
	% ------------------------------------------------------------------------
	% obs.: ao comentar e descomentar a linha que carrega o pacote babel, você
	% pode obter o seguinte erro na compilação:
	%
	% Package babel Error: you haven't defined the language ??? yet.
	%
	% Se isto acontecer, simplesmente recompile que o erro desaparecerá.
	% Pergunte-me por que isso acontece.
	% ------------------------------------------------------------------------
	
	
\end{document}
