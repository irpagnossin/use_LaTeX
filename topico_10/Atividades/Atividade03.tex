\documentclass[a4paper,12pt]{article}
	\usepackage[ansinew]{inputenc}
	\usepackage{ae}
	\usepackage[brazil]{babel}
	\usepackage{xcolor}
	
	\newcommand{\comando}[1]{{\normalfont\textbackslash\color{blue!50!black}#1}}
\newcommand{\pacote}[1]{{\normalfont\sffamily\color{orange!50!black}#1}}
\newcommand{\ambiente}[1]{{\normalfont\sffamily\color{green!50!black}#1}}
\newcommand{\contador}[1]{{\normalfont\sffamily\color{red!50!black}#1}}
\newcommand{\foreign}[1]{\textsl{#1}}
\newcommand\linhabase{\noindent\makebox[0pt][l]{\color{red}\rule{\textwidth}{0.1pt}}}
\begin{document}

	Este arquivo ilustra a utilização do ambiente \ambiente{array} (não confundir com o pacote \pacote{array} da atividade anterior), exclusivo do modo matemático e utilizado para produzir matrizes, por exemplo.

	% A sintaxe do ambiente array é exatamente a mesma de tabular. A única diferença é que array só pode ser utilizada no modo matemático.
	\linhabase\hfill
	\(
		\begin{array}[c]{ccc}
		a_{11} & a_{12} & a_{13} \\
		a_{21} & a_{22} & a_{23} \\
		a_{31} & a_{32} & a_{33}
		\end{array}
	\)%
	\hfill\null
	
	% EXERCÍCIO: posicione os comandos \left( e \right) adequadamente de modo a  produzir uma matriz. Faça o mesmo com \left[ e \right]. Experimente também trocar \( por \[ e \) por \].
	
\end{document}