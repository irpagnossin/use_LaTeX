\documentclass[a4paper,12pt]{article}
	\usepackage[ansinew]{inputenc}
	\usepackage{ae}	
	\usepackage[brazil]{babel}
	\usepackage{xcolor}
	\usepackage{calc}
	\newcommand{\comando}[1]{{\normalfont\textbackslash\color{blue!50!black}#1}}
\newcommand{\pacote}[1]{{\normalfont\sffamily\color{orange!50!black}#1}}
\newcommand{\ambiente}[1]{{\normalfont\sffamily\color{green!50!black}#1}}
\newcommand{\contador}[1]{{\normalfont\sffamily\color{red!50!black}#1}}
\newcommand{\foreign}[1]{\textsl{#1}}
\newcommand\linhabase{\noindent\makebox[0pt][l]{\color{red}\rule{\textwidth}{0.1pt}}}
	
	% O pacote tabularx calcula automaticamente a largura das colunas de modo a satisfazer a extensão horizontal da tabela requerida.
	\usepackage{tabularx}
	
	
	\newcommand\tablecontent{%
		\hline
		Uma célula & Outra célula \tabularnewline
		\hline
		Mai uma    & ... e outra  \tabularnewline
		\hline
	}
	
\begin{document}

	Como você pôde perceber, o \LaTeX\ ajusta automaticamente o tamanho da tabela ao conteúdo dela. Esta solução é geralmente suficiente, mas eventualmente você quererá definir a extensão horizontal da tabela. Para isso existe o ambiente \ambiente{tabular*}, que recebe um argumento a mais, a largura da tabela:	
	
	\medskip
	\noindent
	\begin{tabular*}{\textwidth}{|l|l|}
		\tablecontent
	\end{tabular*}
	\smallskip
	
	Mas isto não influi na largura das \emph{células}, conforme ilustra o exemplo acima. De fato, para produzir o resultado desejado é preciso definir também a largura de cada uma das células, através da opção \texttt{p\{\textrm{\textit{comp}}\}}, e isto requer alguns cálculos envolvendo comprimentos parametrizados da tabela, como \comando{tabcolsep} (a meia-distância entre duas células) e \comando{arrayrulewidth} (a largura das linhas verticais), além da quantidade de colunas da tabela:
	
	\newlength\colwidth
	\setlength\colwidth{0.5\textwidth-2\tabcolsep-1.5\arrayrulewidth}
	
	\medskip
	\noindent
	\begin{tabular*}{\textwidth}{|p{\colwidth}|p{\colwidth}|}
		\tablecontent
	\end{tabular*}
	\smallskip
	
	Obviamente preocupar-se com essas contas é desagradável e pode tornar-se complicado se as larguras das colunas não forem iguais.
	
	Felizmente você não precisa se preocupar com isso... basta usar o pacote \pacote{tabularx}, que define o ambiente \ambiente{tabularx}. Sua sintaxe é exatamente a mesma do \ambiente{tabular*}:
	
	\medskip
	\noindent
	\begin{tabularx}{\textwidth}{|X|X|}
		\tablecontent
	\end{tabularx}
	\smallskip

	A diferença é a opção \texttt{X}: ao usá-la o \LaTeX\ calculará automaticamente a largura das colunas. Na verdade, \texttt{X} equivale a \texttt{p\{\textrm{\textit{comp}}\}}, só que agora \textit{comp} é calculado automaticamente.

\end{document}