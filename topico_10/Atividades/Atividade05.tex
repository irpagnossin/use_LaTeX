\documentclass[a4paper,12pt]{article}
	\usepackage[ansinew]{inputenc}
	\usepackage{ae}
	\usepackage[brazil]{babel}
	\usepackage{xcolor}
	\usepackage{multirow} % <-- Define o comando \multirow.
	
	\newcommand{\comando}[1]{{\normalfont\textbackslash\color{blue!50!black}#1}}
\newcommand{\pacote}[1]{{\normalfont\sffamily\color{orange!50!black}#1}}
\newcommand{\ambiente}[1]{{\normalfont\sffamily\color{green!50!black}#1}}
\newcommand{\contador}[1]{{\normalfont\sffamily\color{red!50!black}#1}}
\newcommand{\foreign}[1]{\textsl{#1}}
\newcommand\linhabase{\noindent\makebox[0pt][l]{\color{red}\rule{\textwidth}{0.1pt}}}
	
	\setlength\parskip\baselineskip
\begin{document}
	
	Este arquivo ilustra a utilização do comando \comando{multirow}, definido pelo pacote \pacote{multirow}, que facilita a unificação de células de uma mesma \emph{coluna}.

	% O pacote multirow define o comando \multirow, que facilita a união de células verticais. Sem ele você teria de utilizar o comando \raisebox para SIMULAR esta união. A sintaxe do comando \multirow é a seguinte:
	%
	% \multirow{número de linhas}{largura}{conteúdo}
	
	\linhabase\hfill
	\begin{tabular}[c]{|c|c|c|}
		\hline
		\multirow{2}{10em}{\centering Células 11 e 21} & Célula 12 & Célula 13 \\
		\cline{2-3}
		                                               & C. 22     & C. 23     \\
		\hline
		\multirow{2}{10em}{\centering Células 31 e 41} & Célula 32 & Célula 33 \\
		\cline{2-3}
		                                               & 42        & 43        \\
		\hline
	\end{tabular}%
	\hfill\null
			
\end{document}