\documentclass[a4paper,12pt]{article}
	\usepackage[ansinew]{inputenc}
	\usepackage{ae}
	\usepackage[brazil]{babel}
	\usepackage{xcolor}
	\usepackage{multirow} % <-- Define comandos como \toprule e \midrule, além de melhorar a aparência de tabelas.
	\usepackage{booktabs}
		
	\newcommand{\comando}[1]{{\normalfont\textbackslash\color{blue!50!black}#1}}
\newcommand{\pacote}[1]{{\normalfont\sffamily\color{orange!50!black}#1}}
\newcommand{\ambiente}[1]{{\normalfont\sffamily\color{green!50!black}#1}}
\newcommand{\contador}[1]{{\normalfont\sffamily\color{red!50!black}#1}}
\newcommand{\foreign}[1]{\textsl{#1}}
\newcommand\linhabase{\noindent\makebox[0pt][l]{\color{red}\rule{\textwidth}{0.1pt}}}
	
	\setlength\parskip\baselineskip	
\begin{document}

	Este arquivo ilustra a utilização do pacote \pacote{booktabs}, que melhora a aparência de tabelas tanto por modificar o espaçamento vertical entre as linhas como por definir comandos como \comando{toprule}, \comando{midrule} e \comando{bottomrule}, que imprimem linhas de espessuras diferentes, cada qual apropriada para um local diferente na tabela.

	% Ao invés de \hline e \cline, procure utilizar os comandos \toprule, \midrule e/ou \cmidrule, e \bottomrule, definidos pelo pacote booktabs. Eles produzem linhas de espessura diferente que melhoram a aparência da tabela. Além disso, o pacote melhora o espaçamento entre as linhas.

	\linhabase\hfill
	\begin{tabular}[c]{ccc}
		\toprule
		\multirow{4}{5em}{\centering União\\vertical} & Célula 12 & Célula 13 \\
		\cmidrule{2-3}
		                         & C. 22     & C. 23     \\
		\cmidrule{2-2}
		                         & 32        & 33        \\
		                         & 42        & 43        \\
		\bottomrule
	\end{tabular}%
	\hfill\null
		
	% Dicas para construir uma tabela inteligível, segundo S. Fear, em "Publication quality tables in LaTeX" (arquivo booktabs.pdf no CTAN (Comprehensive TeX Archive Network, www.ctan.org):
	%
	% 1) NUNCA utilize linhas verticais;
	% 2) NUNCA utilize linhas duplas;
	% 3) Coloque unidades de medidas no título da coluna;
	% 4) Não utilize " ou "idem" para indicar uma repetição. Ao invés disso, deixe a célula vazia ou repita a anterior.
			
	Compare com a mesma tabela, sem utilizar \pacote{booktabs} (ie, sem utilizar os comandos \comando{toprule} e cia):

	\linhabase\hfill
	\begin{tabular}[c]{ccc}
		\hline
		\multirow{4}{5em}{\centering União\\vertical} & Célula 12 & Célula 13 \\
		\cline{2-3}
		                         & C. 22     & C. 23     \\
		\cline{2-2}
		                         & 32        & 33        \\
		                         & 42        & 43        \\
		\hline
	\end{tabular}%
	\hfill\null
				
\end{document}