\documentclass[a4paper,12pt]{article}
	\usepackage[ansinew]{inputenc}
	\usepackage{ae}
	\usepackage[brazil]{babel}
	\usepackage{xcolor}
	\usepackage{colortbl} % <-- Permite o uso de cores em tabelas
	
	\newcommand{\comando}[1]{{\normalfont\textbackslash\color{blue!50!black}#1}}
\newcommand{\pacote}[1]{{\normalfont\sffamily\color{orange!50!black}#1}}
\newcommand{\ambiente}[1]{{\normalfont\sffamily\color{green!50!black}#1}}
\newcommand{\contador}[1]{{\normalfont\sffamily\color{red!50!black}#1}}
\newcommand{\foreign}[1]{\textsl{#1}}
\newcommand\linhabase{\noindent\makebox[0pt][l]{\color{red}\rule{\textwidth}{0.1pt}}}
	
	\setlength\parskip\baselineskip	
\begin{document}

	Este arquivo ilustra a utilização de cores em tabelas através do pacote \pacote{colortbl}. Basicamente são três os comandos úteis: \comando{rowcolor}, para atribuir cor a toda uma linha (exemplo 1); \comando{columncolor} (exemplo 2), para toda uma coluna; e \comando{cellcolor} (exemplo 3), para definir a cor de apenas uma tabela. Há muitas outras opções... não deixe de ver a documentação do pacote.

	%--------------------------------------------
	% EXEMPLO 1
	%--------------------------------------------
	% O comando \rowcolor{cor} recebe como argumento obrigatório uma cor, segundo a sintaxe do pacote color ou xcolor. Ele deve ser colocado logo no início da linha que ser quer colorir.
	
	\linhabase\hfill
	\begin{tabular}[c]{ccc}
		\rowcolor{red!50}     Célula 11 & Célula 12 & Célula 13 \\
		\rowcolor{green!50}   Célula 21 & Célula 22 & Célula 23 \\
		\rowcolor{blue!50}    Célula 31 & Célula 32 & Célula 33 \\
		\rowcolor{yellow!50}  Célula 41 & Célula 42 & Célula 43 \\
		\rowcolor{magenta!50} Célula 51 & Célula 52 & Célula 53 \\
		\rowcolor[gray]{1.0}  Célula 61 & Célula 62 & Célula 63 \\
		\rowcolor[gray]{0.8}  Célula 71 & Célula 72 & Célula 73 \\		
		\rowcolor[gray]{0.6}  Célula 81 & Célula 82 & Célula 83 \\
		\rowcolor{black!60}   Célula 91 & Célula 92 & Célula 93 % black!60 é o mesmo que [gray]{0.4}
	\end{tabular}%
	\hfill\null
	
	%--------------------------------------------
	% EXEMPLO 2
	%--------------------------------------------
	% O comando \columncolor{cor} colore toda uma coluna. Você deve utilizá-lo junto com a opção >{} do pacote array (não é necessário carregá-lo pois o colortbl faz isso).
	
	\linhabase\hfill
	\begin{tabular}[c]{
		>{\columncolor{red!50}}     c
		>{\columncolor{green!50}}   c
		>{\columncolor{blue!50}}    c
		>{\columncolor{yellow!50}}  c
		>{\columncolor{magenta!50}} c
		>{\columncolor{black!50}}   c}
	
		Col. 1 & Col. 2 & Col. 3 & Col. 4 & Col. 5 & Col. 6\\
		Col. 1 & Col. 2 & Col. 3 & Col. 4 & Col. 5 & Col. 6\\
		Col. 1 & Col. 2 & Col. 3 & Col. 4 & Col. 5 & Col. 6\\
		Col. 1 & Col. 2 & Col. 3 & Col. 4 & Col. 5 & Col. 6\\
		Col. 1 & Col. 2 & Col. 3 & Col. 4 & Col. 5 & Col. 6
	\end{tabular}%
	\hfill\null
	
	%--------------------------------------------
	% EXEMPLO 3
	%--------------------------------------------	
	% Finalmente, o comando \cellcolor{cor} torna colorida apenas a célula que contém esta instrução, como abaixo.
	
	\linhabase\hfill
	\begin{tabular}[c]{ccc}	
		\cellcolor{red!50} Col. 1  &                    Col. 2 & \cellcolor{green!50}  Col. 3 \\
		                   Col. 1  & \cellcolor{blue!50}Col. 2 &                       Col. 3 \\
		\cellcolor{cyan!50}Col. 1 &                     Col. 2 & \cellcolor{magenta!50}Col. 3
	\end{tabular}%
	\hfill\null
\end{document}
