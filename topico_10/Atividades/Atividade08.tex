\documentclass[a4paper,12pt]{article}
	\usepackage[ansinew]{inputenc}
	\usepackage{ae}	
	\usepackage[brazil]{babel}
	\usepackage{lipsum}
	\usepackage{xcolor}
	\usepackage{booktabs}
	
	\newcommand{\comando}[1]{{\normalfont\textbackslash\color{blue!50!black}#1}}
\newcommand{\pacote}[1]{{\normalfont\sffamily\color{orange!50!black}#1}}
\newcommand{\ambiente}[1]{{\normalfont\sffamily\color{green!50!black}#1}}
\newcommand{\contador}[1]{{\normalfont\sffamily\color{red!50!black}#1}}
\newcommand{\foreign}[1]{\textsl{#1}}
\newcommand\linhabase{\noindent\makebox[0pt][l]{\color{red}\rule{\textwidth}{0.1pt}}}
	
	% Os parâmetros \topfraction e \bottomfraction representam o tamanho máximo que um objeto flutuante pode ter quando colocado no topo da página ou na base dela, respectivamente. Os valores padrão são 0.7 (70% da altura do texto) e 0.3 (30% da altura do texto), respectivamente.
	\renewcommand\topfraction{0.5}
	\renewcommand\bottomfraction{0.5}
	
\begin{document}

	% O ambiente table atribui à(s) caixa(s) contida(s) nele a capacidade de "flutuar" pelo texto. Isto é, o LaTeX pode reposicionar o elemento do texto (tabela, usualmente) que está contido num ambiente table de modo a satisfazer vários parâmetros. Mas em última instância este reposicionamento ocorre porque a(s) caixa(s) em questão não cabem onde foram inseridas!

	\lipsum[1]\lipsum[1]

	\begin{table}[htbp]
		\centering
		\raisebox{-\topfraction\textheight}[0pt][0pt]{\color{red}\rule{2pt}{\topfraction\textheight}}%
		%\raisebox{-\bottomfraction\textheight}[0pt][0pt]{\color{red}\rule{2pt}{\bottomfraction\textheight}}%
		\begin{tabular}[t]{ccc}
			\toprule
			Coluna 1  & Coluna 2  & Coluna 3  \\
			\midrule
			Célula 21 & Célula 22 & Célula 23 \\
			Célula 21 & Célula 22 & Célula 23 \\
			Célula 21 & Célula 22 & Célula 23 \\
			Célula 21 & Célula 22 & Célula 23 \\
			Célula 21 & Célula 22 & Célula 23 \\ 
			Célula 21 & Célula 22 & Célula 23 \\ 
			Célula 21 & Célula 22 & Célula 23 \\
			Célula 21 & Célula 22 & Célula 23 \\
			Célula 21 & Célula 22 & Célula 23 \\
			Célula 21 & Célula 22 & Célula 23 \\ 
%			Célula 21 & Célula 22 & Célula 23 \\ % <-- Até aqui vai para "h"
%			Célula 21 & Célula 22 & Célula 23 \\ % <-- A partir daqui vai para "t" ou "b"
%			Célula 21 & Célula 22 & Célula 23 \\
%			Célula 21 & Célula 22 & Célula 23 \\
%			Célula 21 & Célula 22 & Célula 23 \\
%			Célula 21 & Célula 22 & Célula 23 \\ % <-- A partir daqui vai para "p"
			\bottomrule
		\end{tabular}
		
		\caption{o comando \comando{label}, qua atribui um nome (ou rótulo) à tabela, deve sempre ser utilizado \emph{após} o comando \comando{caption} ou dentro de seu argumento.}
		\label{tab:rotulo}
	\end{table}
	
	\lipsum[4-7]
	
\end{document}
