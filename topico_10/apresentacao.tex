%-------------------------------------------------------------------------------
% Autores: I. R. Pagnossin e Centro de Ensino e Pesquisa Aplicada.
%
% Este material é parte integrante do curso "Usando LaTeX; pensando em TeX" e é
% distribuido pelos autores segundo a licença Creative Commons 2.5 Brasil
% (atribuição/não-comercial/redistribuição segundo a mesma licença).
%
% This material is part of the course "Usando LaTeX; pensando em TeX".
% It is distributed according to the license Creative Commons 2.5 Brazil
% (attribution/non-comercial use/share alike the same license).
%-------------------------------------------------------------------------------
\newif\ifhandout
%\handouttrue  % Descomente se for para gerar a versão para IMPRESSÃO.
\handoutfalse % Descomente se for para gerar a versão para APRESENTAÇÃO

%-------------------------------------------------------------------------------
\ifhandout
  \documentclass[handout,10pt]{beamer}
  \mode<handout>
\else
	\documentclass[10pt,hyperref={pdfpagelabels=false}]{beamer}
	\mode<presentation>
\fi

	\usepackage[utf8]{inputenc}
	\usepackage[brazil]{babel}	
	\usepackage{graphicx}
	\usepackage{listings}
	\usepackage{amsmath}
	\usepackage{tikz}	
	\usepackage[squaren]{SIunits}	
	\usepackage{fancybox}
	\usepackage{array}
	\usepackage{colortbl}		
	\usepackage{bookman}
	\usepackage{fancybox}
	\usepackage{booktabs}
	\usepackage{helvet}
	\usepackage{lipsum}
	\usepackage{tabularx}
	\usepackage{calc}
	\usepackage{colortbl}
	\usepackage{multirow}
		
	\ifhandout
		\usepackage{pgfpages}
		\pgfpagesuselayout{2 on 1}[a4paper,border shrink=5mm]
	\fi

	% Bibliotecas TikZ e PGF necessárias
	\usetikzlibrary{shapes.symbols}
	\usepgflibrary{shapes.misc}
	\usetikzlibrary{calc}
	

	% Path das figuras, relativo a esta pasta.
	\graphicspath{{../arquivos_comuns/figuras/}{./figuras/}}

	% Modelo da apresentação	
	\usetheme{Frankfurt}
	\usefonttheme{serif,structurebold}
	%\setbeamercovered{transparent}		
		
	% Metadados do arquivo PDF.
	\hypersetup{
		pdftitle={Tabelas},
		pdfauthor={Dr. Ivan R. Pagnossin},
		pdfsubject={LaTeX},
		pdfkeywords={TeX,LaTeX}
	}

	% Título, autores e instituição.
	\title{Tabelas}
	\author{\textbf{Prof.:} Ivan R. Pagnossin\and \textbf{Tutora:} Juliana Giordano}
	\institute{%
		Coordenadoria de Tecnologia da Informação\\
		Centro de Ensino e Pesquisa Aplicada}
	\logo{\includegraphics[width=0.25\textwidth]{LogotipoCursoLaTeX_v3_pequeno}}
	\date{}
	
	\newcommand{\scalefactor}{7}
	\tikzset{box/.style={color=#1,thick},box/.default=red}
	\tikzset{letter/.style={color=#1,inner sep=0pt},letter/.default=gray!50}
	\tikzset{baseline/.style={color=#1,thin},baseline/.default=blue}
		
	\newsavebox{\meuparbox}
	
\begin{document}
%-------------------------------------------------------------------
\begin{frame}[c,label=titulo]
	\centering	
	
	\includegraphics[width=0.8\textwidth]{LogotipoCursoLaTeX_v2}

	\titlepage
\end{frame}
\end{document}
%-------------------------------------------------------------------
\logo{} % <-- O logotipo não aparecerá mais a partir daqui.
\setbeamertemplate{background canvas}{%
		\includegraphics[width=\paperwidth,height=\paperheight,keepaspectratio=false]{leao-pensador-wattermark.png}
}
%-------------------------------------------------------------------
\newsavebox{\tableboxA}
\begin{lrbox}{\tableboxA} % <-- O ambiente lrbox equivale a \savebox
\begin{LaTeXcode}			
\begin{tabular}[t]{cc}
  \hline
  Célula 11 & Célula 12 \\
  C. 21     & C. 22     \\
  31        & 32        \\
  \hline
\end{tabular}
\end{LaTeXcode}
\end{lrbox}

\newsavebox{\tableboxB}
\begin{lrbox}{\tableboxB}
\begin{LaTeXcode}				
\begin{tabular}[b]{cc}
  \hline
  Célula 11 & Célula 12 \\
  C. 21     & C. 22     \\
  31        & 32        \\
  \hline
\end{tabular}
\end{LaTeXcode}
\end{lrbox}

\newlength{\heightoftableboxB}
\settoheight{\heightoftableboxB}{\usebox{\tableboxB}}

\newsavebox{\tableboxC}
\begin{lrbox}{\tableboxC} 
\begin{LaTeXcode}				
\begin{tabular}[c]{cc}
  \hline
  Célula 11 & Célula 12 \\
  C. 21     & C. 22     \\
  31        & 32        \\
  \hline
\end{tabular}
\end{LaTeXcode}
\end{lrbox}

\newsavebox{\tableboxD}
\begin{lrbox}{\tableboxD}
\begin{LaTeXcode}				
\begin{tabular}[c]{ll}
  \hline
  Célula 11 & Célula 12 \\
  C. 21     & C. 22     \\
  31        & 32        \\
  \hline
\end{tabular}
\end{LaTeXcode}
\end{lrbox}

\newsavebox{\tableboxE}
\begin{lrbox}{\tableboxE}
\begin{LaTeXcode}				
\begin{tabular}[c]{rr}
  \hline
  Célula 11 & Célula 12 \\
  C. 21     & C. 22     \\
  31        & 32        \\
  \hline
\end{tabular}
\end{LaTeXcode}
\end{lrbox}

\section{Tabelas}
\subsection{Estrutura básica}
\begin{frame}[fragile]
	\frametitle{A estrutura básica}
	
	\begin{block}{}
		\begin{center}
			\cs{begin}\verb|{tabular}[|\textit{alinh. \textbf{tabela}}%
			\verb|]{|\textit{alinh. \textbf{colunas}}\verb|}|
		\end{center}
	
		\medskip
	
		\footnotesize
		\begin{tabular}{ll}
			\textit{alinh. tabela:}&
				\texttt{\alert<2>{t}} (topo),
				\texttt{\alert<3>{b}} (base) ou
				\texttt{\alert<4->{c}} (centro; padrão)\\
			\textit{alinh. colunas:}&
				\texttt{\alert<5>{l}} (esquerda),
				\texttt{\alert<6>{r}} (direita) ou
				\texttt{\alert<2-4>{c}} (padrão)
		\end{tabular}
	\end{block}
		
	\begin{columns}
		\column{0.50\textwidth}
		
		\begin{atividade}<2->{1}
			\hspace{-1em}\onslide<2>{\usebox{\tableboxA}}%
			\llap{\onslide<3>{\usebox{\tableboxB}}}%
			\llap{\onslide<4>{\usebox{\tableboxC}}}%
			\llap{\onslide<5>{\usebox{\tableboxD}}}%
			\llap{\onslide<6->{\usebox{\tableboxE}}}		
		\end{atividade}
		
		
		\column{0.45\textwidth}
		
		\begin{uncoverenv}<2->
			\rule[-0.5\heightoftableboxB]{0pt}{\heightoftableboxB}%
			{\color{red}\hrulefill}
			\onslide<2>{\begin{tabular}[t]{cc}
				\hline
				Célula 11 & Célula 12 \\
				C. 21     & C. 22     \\
				31        & 32        \\
				\hline
			\end{tabular}}%
			\llap{\onslide<3>{\begin{tabular}[b]{cc}
				\hline
				Célula 11 & Célula 12 \\
				C. 21     & C. 22     \\
				31        & 32        \\
				\hline
			\end{tabular}}}%
			\llap{\onslide<4>{\begin{tabular}[c]{cc}
				\hline
				Célula 11 & Célula 12 \\
				C. 21     & C. 22     \\
				31        & 32        \\
				\hline
			\end{tabular}}}%
			\llap{\onslide<5>{\begin{tabular}[c]{ll}
				\hline
				Célula 11 & Célula 12 \\
				C. 21     & C. 22     \\
				31        & 32        \\
				\hline
			\end{tabular}}}%
			\llap{\onslide<6->{\begin{tabular}[c]{rr}
				\hline
				Célula 11 & Célula 12 \\
				C. 21     & C. 22     \\
				31        & 32        \\
				\hline
			\end{tabular}}}
			{\color{red}\hrulefill}
		\end{uncoverenv}		
	\end{columns}
	
		\begin{center}
		\begin{actionenv}<2->
			\begin{tikzpicture}
			\node [rectangle,rounded corners=1mm,minimum size=7mm,very thick,
				draw=red!30, top color=red!20,bottom color=white!20]
				{\& separa colunas; \textbackslash\textbackslash\ separa linhas.};
			\end{tikzpicture}
		\end{actionenv}	
	\end{center}
	
\end{frame}
%-------------------------------------------------------------------
\subsection{Linhas horizontais e matrizes}
\begin{frame}
	\frametitle{Linhas horizontais e matrizes}

	\begin{atividade}<1->{2}
		\hrulefill
		\begin{tabular}[c]{ccc}
			\cline{1-3}
			Célula 11 & Célula 12 & Célula 13 \\
			\cline{1-2}
			C. 21     & C. 22     & C. 23     \\
			\cline{1-1}\cline{3-3}
			31        & 32        & 33
		\end{tabular}%
		\hrulefill		
	\end{atividade}
	
	\vfill
	
	\begin{columns}
	\column{0.45\textwidth}
	\begin{atividade}<2->{3}\small
		O ambiente \ambiente{array} é muito similar ao \ambiente{tabular}, exceto 
		que deve ser utilizado no modo matemático.		
		\[
		\left(\begin{array}[c]{ccc}
		a_{11} & a_{12} & a_{13} \\
		a_{21} & a_{22} & a_{23} \\
		a_{31} & a_{32} & a_{33}
		\end{array}\right)
		\]	
	\end{atividade}

	\vfill

	\column{0.45\textwidth}
	\begin{block}<3->{Exercício}
		Reproduza a expressão matricial:
		\[
		\left(\begin{array}{c}
			x\,' \\ y\,'
		\end{array}\right) =
		\left(\begin{array}{cc}
			a_{xx} & a_{xy} \\ a_{yx} & a_{yy}
		\end{array}\right)
		\left(\begin{array}{c}
			x \\ y
		\end{array}\right)
		\]
	\end{block}
	\end{columns}
	
\end{frame}
%-------------------------------------------------------------------
\subsection{Uniões de células}
\begin{frame}
	\frametitle{Uniões de células (\foreign{merging})}

	\begin{atividade}<1->{4}
		A união de células na horizontal pode ser facilmente obtida através do 
		comando \cs{multicolumn}.
		
		\medskip
		
		\hrulefill
		\begin{tabular}[c]{|c|c|c|}
			\hline
			\multicolumn{2}{|c|}{Células 11 e 12} & Célula 13 \\
			\hline
			C. 21     & \multicolumn{2}{c|}{Células 22 e 23} \\
			\hline
			31        & 32        & \multicolumn{1}{c|}{Célula 31} \\
			\hline
		\end{tabular}%
		\hrulefill	
	\end{atividade}
	
	\begin{atividade}<2->{5}
		Utilize o pacote \pkg{multirow} para unir células na vertical.
		
		\medskip
		
		\hrulefill
		\begin{tabular}[c]{|c|c|c|}
			\hline
			\multirow{2}{10em}{\centering Células 11 e 21} & Célula 12 & Célula 13 \\
			\cline{2-3}
			                                               & C. 22     & C. 23     \\
			\hline
			\multirow{2}{10em}{\centering Células 31 e 41} & Célula 32 & Célula 33 \\
			\cline{2-3}
			                                               & 42        & 43        \\
			\hline
		\end{tabular}%
		\hrulefill
	\end{atividade}

\end{frame}
%-------------------------------------------------------------------
\subsection{O pacote booktabs}
\begin{frame}	
	\frametitle{O pacote \pkg{booktabs}}

	\begin{atividade}<1->{6}
		\noindent\hrulefill
		\begin{tabular}[c]{ccc}
			\toprule
			Célula 11 & Célula 12 & Célula 13 \\
			\midrule
			C. 21 & C. 22     & C. 23     \\			
			21  & 32        & 33        \\
			\bottomrule
		\end{tabular}%
		\hrulefill	
	\end{atividade}
	
	\small
	\uncover<2->{Dicas:}
	\begin{itemize}
		\item<2-> Use o pacote \pkg{booktabs};
		\item<3-> \emph{Nunca} utilize linhas verticais;
		\item<4-> \emph{Nunca} utilize linhas duplas;
		\item<5-> Coloque unidades de medidas no título da coluna;
		\item<6-> Não utilize "\ para indicar uma repetição. Ao invés
			disso, deixe a célula vazia ou repita a anterior.
	\end{itemize}
	
\end{frame}

\ifhandout\else
{
	\logo{\includegraphics[width=0.25\textwidth]{LogotipoCursoLaTeX_v3_pequeno}}
	\setbeamertemplate{background canvas}{}
	\againframe{titulo} % Reapresenta a página inicial.
}
\fi
%-------------------------------------------------------------------
%-------------------------------------------------------------------
%-------------------------------------------------------------------
\appendix
\section{Respostas}
\begin{frame}[fragile,label=respostas]
	\frametitle{Respostas dos exercícios}

	\footnotesize

	\begin{enumerate}
	\item
	\item%
	\end{enumerate}

\end{frame}
\end{document}