\documentclass[a4paper,10pt,twocolumn,landscape]{article}
	\usepackage[ansinew]{inputenc}
	\usepackage{ae}
	\usepackage[brazil]{babel}
	\usepackage{xcolor}
	\usepackage{calc}
	\usepackage{wallpaper}
	\usepackage[margin=2cm]{geometry}
	\usepackage{SIunits}
	\usepackage{amsmath}
	\usepackage{paralist}

	% Minhas definições
	\newcounter{exercicio}
\newenvironment{exercicio}{%
	\refstepcounter{exercicio}%
	\penalty-200
	\noindent\colorbox{blue!60!black}{\makebox[\columnwidth-\fboxsep*2][c]{\textbf{\color{white}Exercício~\theexercicio}}}\smallskip
}{\par\medskip}
\newcommand{\foreign}[1]{\textsl{#1}}
\newcommand{\digite}[1]{{\fontfamily{cmss}\fontseries{bx}\selectfont#1}}	
\newcommand{\cs}[1]{{\normalfont\textbackslash\color{blue!50!black}#1}}
\newcommand{\pkg}[1]{{\normalfont\sffamily\color{orange}#1}}
\newcommand{\env}[1]{{\normalfont\sffamily\color{green!50!black}#1}}
\let\comando=\cs
\let\package=\pkg
\let\ambiente=\env

	% Papel de parede
	\CenterWallPaper{1}{../figuras/leao-pensador-wattermark.png}

	% Configuração extra: espaço entre as colunas.
	\setlength{\columnsep}{2cm}

	\DeclareMathOperator{\sen}{sen}

	\hyphenation{de-ter-mi-na-ram}

\begin{document}

\begin{center}\includegraphics[width=0.8\columnwidth]{../figuras/LogotipoCursoLaTeX_v2}\end{center}

	\section*{Expressões matemáticas no \LaTeX\ --- Série 1}

	Reproduza todas as expressões matemáticas abaixo num arquivo \LaTeX\ e, quando terminado, envie-o para o professor através do moodle. Você pode utilizar qualquer comando ou caráter especial para alternar entre os modos parágrafo e matemático, mas atenção para o fato de que algumas expressões têm números (ou nomes?), o que limita a sua escolha nesses casos.
	
	\bigskip
	
	%---------------------------------------------------------
	\begin{exercicio}
	
		\[
		\left(a + b\right)^3 = a^3 + 3 a^2 b + 3 a b^2 + b^3
		\]	
	\end{exercicio}
	
	%---------------------------------------------------------
	\begin{exercicio}
	
		\begin{equation}\label{prod:notavel}
		ax^2 + bx + c = \left(x - x_0\right)\left(x - x_1\right)
		\end{equation}
	\end{exercicio}
	

	%---------------------------------------------------------
	\begin{exercicio}
	
		\[
		\frac{a^2 - b^2}{a + b} = a - b
		\]
	\end{exercicio}
	

	%---------------------------------------------------------
	\begin{exercicio}
	
		\[
		\frac{1}{1 + \frac{1}{1 + \frac{1}{1 + \cdots}}}
		\]
	\end{exercicio}
	
	
	%---------------------------------------------------------
	\begin{exercicio}
	
		\[
		\sqrt{2xy - x^2 - y^2} = i(x-y)
		\]
	\end{exercicio}
	
	
	%---------------------------------------------------------
	\begin{exercicio}
	
		\[
		\sqrt{a^2 + \sqrt{a^2 + \sqrt{a^2 + \cdots}}}
		\]
	\end{exercicio}
	
	
	%---------------------------------------------------------
	\begin{exercicio}	
	
		\[
		\int\limits_{-1}^2 x^4\, dx = \left.\frac{x^5}{5}\right|_{-1}^2 = \frac{33}{5}
		\]
	
	\bigskip		
	\noindent	obs.: utilize \verb|\left.| e \verb!\right|! para produzir a barra vertical.
	\end{exercicio}

	%---------------------------------------------------------

	
	\hspace*{\stretch{1}}\rule{0.5\columnwidth}{0.1pt}\hspace*{\stretch{1}}
	\medskip

	Para fazer os próximos exercícios você precisará consultar a lista de símbolos matemáticos do livro \textit{Introdução ao \LaTeXe}, páginas 62--68. Além disso, tenha em mente o seguinte:
	\begin{compactitem}
		\item ``\(\cos\)'' é produzido com \verb|\cos|;
		\item ``\(\sen\)'' é produzido com \verb|\sen| (não use \verb|\sin|);
		\item ``\(\ln\)'' é produzido com \verb|\ln|.
	\end{compactitem}

	\bigskip
	%---------------------------------------------------------
	\begin{exercicio} Construa a seguinte frase:
	
		\centering A área do círculo de raio \(r\) é \(\pi r^2\) e seu perímetro, \(2\pi r\).
	\end{exercicio}
	
	
	%---------------------------------------------------------
	\begin{exercicio}\label{ex:cos}
	
		\[
		\cos\left(\frac{\pi}{4}\right) = \frac{\sqrt{2}}{2}
		\]
	\end{exercicio}
	

	%---------------------------------------------------------
	\begin{exercicio}\label{ex:sin}
	
		\begin{equation}\label{eq:sen(A+B)}
		\sen\left(\theta+\varphi\right) = \sen\theta \cos\varphi + \cos\theta \sen\varphi
		\end{equation}
	\end{exercicio}
	
		
	%---------------------------------------------------------
	\begin{exercicio}	
	
		\[
		\sen^2 \phi + \cos^2 \phi = 1
		\]		
	\end{exercicio}
	
	
	%---------------------------------------------------------
	\begin{exercicio}
	
		\[
		e^{i\theta} = \cos(\theta) + i\cdot\sen(\theta)
		\]
	\end{exercicio}
	
	
	%---------------------------------------------------------
	\begin{exercicio} \textbf{A fórmula de Báscara.}
	
		A equação~\ref{eq:bascara} determina as abscissas $x_+$ e $x_-$ para as quais a parábola \(y = a x^2 + bx + c\), com \(a \ne 0\), cruza o eixo $y = 0$. Isto é, \eqref{eq:bascara} determina as raizes dessa equação:
		\begin{equation}\label{eq:bascara}
		x_\pm = \frac{-b \pm \sqrt{b^2 - 4ac}}{2a}.
		\end{equation}
		
	\bigskip	
	\noindent obs. 1: você deve gerar as referências à equação com \verb|\label| e \verb|\ref|.\\
	obs. 2: não se esqueça do ponto-final após a equação.
	\end{exercicio}
	
	
	%---------------------------------------------------------
	\begin{exercicio} \textbf{A fórmula de Euler.}
	
		\[
		\sum_{n = 1}^\infty \frac{1}{n^2} = \frac{\pi^2}{6}
		\]
	\end{exercicio}
	
	
	%---------------------------------------------------------
	\begin{exercicio}\label{ex:ln}
	
		\[
		\ln (1 + x) = x - \frac{x^2}{2} + \frac{x^3}{3} - \cdots, \qquad -1 < x \le 1
		\]
		
	\bigskip
	\noindent obs.: utilize \verb|\qquad| após a reticências.
	\end{exercicio}
	
	
	%---------------------------------------------------------
	\begin{exercicio} Reproduza o exercício de Física abaixo, tomando o cuidado de integrar corretamente o texto com as expressões. Atenção especial deve ser dada às unidades de medidas: reproduza-as fielmente.
	
		\bigskip
	
		\begin{quotation}
			\noindent
			\textbf{Enunciado:}
			\begin{slshape}
				engenheiros da força aérea, em 1946, determinaram a distância Terra-Lua usando um radar. Se o feixe do radar levou \unit{2,56}{\second} para completar a viagem total Terra-Lua-Terra, qual a distância Terra-Lua em \kilo\metre? A velocidade das ondas do radar é \unit{3\times 10^{8}}{\metre\per\second}.
			\end{slshape}
			
			\noindent\textbf{Resposta:} sabendo-se a velocidade das ondas de rádio (\(v\)) e o tempo gasto por elas no percurso (\(\Delta t\)), podemos determinar facilmente a distância Terra-Lua através da definição de velocidade: \(v = \Delta s/\Delta t\).
			
			O tempo gasto pelas ondas de rádio para ir da Terra à Lua é igual a metade do tempo medido, que corresponde ao percurso de ida e volta. Isto é, \(\Delta t = 2,56/2 = \unit{1,28}{\second}\). E como \(v = \unit{3 \times 10^{8}}{\metre\per\second}\), a distância Terra-Lua é
			\[
			\Delta s = v \,\Delta t = 3\times 10^{8} \cdot 1,28 = \unit{3,84 \times 10^{5}}{\kilo\metre}.
			\]
		\end{quotation}
	\end{exercicio}

\end{document}                                       
