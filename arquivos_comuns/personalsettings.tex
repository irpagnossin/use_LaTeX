% Configurações personalizadas do código LaTeX.	
\lstnewenvironment{LaTeXcode}{
	\setlength{\abovecaptionskip}{0pt}	
	\lstset{language=[LaTeX]TeX}
	\lstset{%
		basicstyle=\footnotesize\ttfamily,  % Global
		keywordstyle=\color{blue}\bfseries, % Comandos
		identifierstyle=,                   % Texto
		stringstyle=,                       % Strings 
		commentstyle=\color{gray},          % Comentários
		showstringspaces=false,             % Espaços
		rulecolor=\color{gray},             % Linha da caixa
	}
	\lstset{emph={setlength,includegraphics,psfrag,subfigure},emphstyle={\color{blue}\bfseries}}
}% Abrindo o ambiente.
{}% Fechando o ambiente.
	
\newcommand{\digite}[1]{{\fontfamily{cmss}\fontseries{bx}\selectfont#1}}	
\newcommand{\cs}[1]{{\normalfont\textbackslash\color{blue!50!black}#1}}
\newcommand{\pkg}[1]{{\normalfont\sffamily\color{orange}#1}}
\newcommand{\env}[1]{{\normalfont\sffamily\color{green!50!black}#1}}
\let\comando=\cs
\let\package=\pkg
\let\ambiente=\env
\newcommand{\foreign}[1]{{\textsl{#1}}}


	\newcounter{exercicio}	
	\newenvironment{exercicio}{%
		\refstepcounter{exercicio}%
		\penalty-200
		\noindent\colorbox{blue!60!black}{\makebox[\columnwidth-\fboxsep*2][c]{\textbf{\color{white}Exercício~\theexercicio}}}\smallskip
	}{\par\medskip}
		

\newcommand{\bibtex}{\textsc{Bib}\TeX}

\newenvironment<>{atividade}[1]{%
\begin{actionenv}#2%
\begin{exampleblock}{{Atividade #1}}%
}
{%
\end{exampleblock}%
\end{actionenv}%
}
