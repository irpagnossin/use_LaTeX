%-------------------------------------------------------------------------------
% Autores: I. R. Pagnossin e Centro de Ensino e Pesquisa Aplicada.
%
% Este material é parte integrante do curso "Usando LaTeX; pensando em TeX" e é
% distribuido pelos autores segundo a licença Creative Commons 2.5 Brasil
% (atribuição/não-comercial/redistribuição segundo a mesma licença).
%
% This material is part of the course "Usando LaTeX; pensando em TeX".
% It is distributed according to the license Creative Commons 2.5 Brazil
% (attribution/non-comercial use/share alike the same license).
%-------------------------------------------------------------------------------
\newif\ifhandout
%\handouttrue  % Descomente se for para gerar a versão para IMPRESSÃO.
\handoutfalse % Descomente se for para gerar a versão para APRESENTAÇÃO

%-------------------------------------------------------------------------------
\ifhandout
  \documentclass[handout,10pt]{beamer}
  \mode<handout>
\else
	\documentclass[10pt,hyperref={pdfpagelabels=false}]{beamer}
	\mode<presentation>
\fi

	\usepackage[utf8]{inputenc}
	\usepackage{ae}	
	\usepackage[brazil]{babel}	
	\usepackage{graphicx}
	\usepackage{listings}
	\usepackage{tikz}
	\usepackage{amsmath}
	\usepackage[squaren]{SIunits}
	\usepackage{booktabs}
	\usepackage{fancybox}
	\usepackage{array}
	\usepackage{bookman}
	
	\ifhandout
		\usepackage{pgfpages}
		\pgfpagesuselayout{2 on 1}[a4paper,border shrink=5mm]
	\fi

	\usetikzlibrary{shapes.symbols}

	% Configurações pessoais
	% Configurações personalizadas do código LaTeX.	
\lstnewenvironment{LaTeXcode}{
	\setlength{\abovecaptionskip}{0pt}	
	\lstset{language=[LaTeX]TeX}
	\lstset{%
		basicstyle=\footnotesize\ttfamily,  % Global
		keywordstyle=\color{blue}\bfseries, % Comandos
		identifierstyle=,                   % Texto
		stringstyle=,                       % Strings 
		commentstyle=\color{gray},          % Comentários
		showstringspaces=false,             % Espaços
		rulecolor=\color{gray},             % Linha da caixa
	}
	\lstset{emph={setlength,includegraphics,psfrag,subfigure},emphstyle={\color{blue}\bfseries}}
}% Abrindo o ambiente.
{}% Fechando o ambiente.
	
\newcommand{\digite}[1]{{\fontfamily{cmss}\fontseries{bx}\selectfont#1}}	
\newcommand{\cs}[1]{{\normalfont\textbackslash\color{blue!50!black}#1}}
\newcommand{\pkg}[1]{{\normalfont\sffamily\color{orange}#1}}
\newcommand{\env}[1]{{\normalfont\sffamily\color{green!50!black}#1}}
\let\comando=\cs
\let\package=\pkg
\let\ambiente=\env
\newcommand{\foreign}[1]{{\textsl{#1}}}


	\newcounter{exercicio}	
	\newenvironment{exercicio}{%
		\refstepcounter{exercicio}%
		\penalty-200
		\noindent\colorbox{blue!60!black}{\makebox[\columnwidth-\fboxsep*2][c]{\textbf{\color{white}Exercício~\theexercicio}}}\smallskip
	}{\par\medskip}
		

\newcommand{\bibtex}{\textsc{Bib}\TeX}

\newenvironment<>{atividade}[1]{%
\begin{actionenv}#2%
\begin{exampleblock}{{Atividade #1}}%
}
{%
\end{exampleblock}%
\end{actionenv}%
}


	% Path das figuras, relativo a esta pasta.
	\graphicspath{{../arquivos_comuns/figuras/}{./figuras/}}

	% Modelo da apresentação	
	\usetheme{Frankfurt}
	\usefonttheme{serif,structurebold}
	\setbeamercovered{transparent}
		
	\newcommand{\grupo}[1]{\colorbox{red!50}{#1}}
	\newcommand{\escopo}[1]{\colorbox{yellow!50}{#1}}
		
	% Metadados do arquivo PDF.
	\hypersetup{
		pdftitle={Grupos e escopo de comandos},
		pdfauthor={Dr. Ivan R. Pagnossin},
		pdfsubject={LaTeX},
		pdfkeywords={TeX,LaTeX}
	}

	% Título, autores e instituição.
	\title{Grupos e escopos de comando}
	\subtitle{e o alinhamento do texto}
	\author{\textbf{Prof.:} Ivan R. Pagnossin \and \textbf{Tutora:} Juliana Giordano}
	\institute{%
		Coordenadoria de Tecnologia da Informação\\
		Centro de Ensino e Pesquisa Aplicada}
	\logo{\includegraphics[width=0.25\textwidth]{LogotipoCursoLaTeX_v3_pequeno}}
	\date{}
	
\begin{document}


%-------------------------------------------------------------------
\begin{frame}[c,label=titulo]
	\centering	
	
	\includegraphics[width=0.8\textwidth]{LogotipoCursoLaTeX_v2}

	\titlepage
\end{frame}
%-------------------------------------------------------------------
\logo{} % <-- O logotipo não aparecerá mais a partir daqui.
\setbeamertemplate{background canvas}{%
		\includegraphics[width=\paperwidth,height=\paperheight,keepaspectratio=false]{leao-pensador-wattermark.png}
}

\section{Grupos e escopos}
\subsection{Definição}

\begin{frame}[fragile]
	\frametitle{Grupo}
	\framesubtitle{definição}
	
	\begin{block}<2->{Grupo}
		No \LaTeX, um grupo de elementos é definido através das chaves
	\end{block}\vfill
		
	\onslide<3->{Exemplos:}
	\begin{itemize}
	\item<3-> {\large ab\textcolor{red}{\bfseries\Large\{}cde\textcolor{red}{\bfseries\Large\}}fg}
						\hfill \(\leftarrow\) ``cde'' são os elementos do grupo.
	\item<4-> {\large\textcolor{blue}{\bfseries\Large\{}ab\textcolor{red}{\bfseries\Large\{}cde\textcolor{red}{\bfseries\Large\}}fg\textcolor{blue}{\bfseries\Large\}}}
						\hfill \(\leftarrow\) Um grupo pode conter outro
	\item<5-> {\large\textcolor{blue}{\bfseries\Large\{}ab\textcolor{red}{\bfseries\Large\{}cde\textcolor{blue}{\bfseries\Large\}}fg\textcolor{red}{\bfseries\Large\}}}
						\hfill \(\leftarrow\) \alert<5>{\emph{Não} existe intersecção}
	\item<6-> {\large ab\textcolor{red}{\bfseries\Large\{}\textcolor{red}{\bfseries\Large\}}cdefg}
						\hfill \(\leftarrow\) Grupo vazio
	\item<7-> {\large\textcolor{red}{\bfseries\Large\{}\verb|\LaT|\textcolor{red}{\bfseries\Large\}}\verb|eX|}
						\hfill \(\leftarrow\) Cuidado!
	\end{itemize}
			
	\begin{atividade}<8->{1}
		\verb|a {} b|\quad não é a mesma coisa que\quad \verb|a  b|.
	\end{atividade}
			
\end{frame}
%-------------------------------------------------------------------
\begin{frame}
	\frametitle{Escopo}
	\framesubtitle{Definição}
	
	\begin{block}<1->{Escopo}
		É a região ou intervalo sujeito à ação de uma instrução.
	\end{block}\vfill
	
	\uncover<2->{O escopo de um(a)\dots}
	\begin{itemize}
	\item<2-> \textbf{\dots\ comando ordinário} é seus argumentos
	\item<3-> \textbf{\dots\ ambiente} é tudo entre \cs{begin} e \cs{end}
	\item<4-> \textbf{\dots\ declaração} é tudo o que a segue, até o fim do grupo ao qual ela pertence
	\end{itemize}
\end{frame}
%-------------------------------------------------------------------
\subsection{Escopo de comandos ordinários}
\setlength{\fboxrule}{1pt}
\begin{frame}[fragile]
	\frametitle{O escopo de um comando ordinário}
	\framesubtitle{é seus argumentos}
		
	\begin{atividade}{2}
		\begin{LaTeXcode}
			texto \textbf{texto \textit{texto} texto.}
		\end{LaTeXcode}
	\end{atividade}\vfill
	
	\begin{itemize}
	\item<2-> texto \cs{textbf}\{\grupo{\escopo{texto \cs{textit}\{texto\} texto.}}\}
	\item<3-> texto \cs{textbf}\{texto \cs{textit}\{\grupo{\escopo{texto}}\} texto.\}
	\end{itemize}\vfill
	
	\begin{block}<4->{Exercício 1\hfill \hyperlink{respostas}{\footnotesize\textbf{(resposta)}}}
		Na atividade anterior, modifique a posição de \emph{uma} das chaves de modo que
		apenas os dois ``texto'' centrais fiquem em negrito.
	\end{block}\vfill
	
	\begin{uncoverenv}<2->
		\footnotesize
		Legenda: \grupo{grupo}\quad\escopo{escopo}
	\end{uncoverenv}
	
\end{frame}
%-------------------------------------------------------------------
\subsection{Escopo de declarações}
\begin{frame}[fragile]
	\frametitle{O escopo de uma declaração}
	\framesubtitle{é tudo o que a segue, até o fim do grupo ao qual percence}
	
	\begin{atividade}{3}
		\begin{LaTeXcode}
			texto \bfseries {texto \itshape texto} texto.
		\end{LaTeXcode}
	\end{atividade}\vfill
	
	\begin{itemize}
	\item<2-> \grupo{texto \cs{bfseries} \escopo{\{texto \cs{itshape} texto\} texto.}}
	\item<3-> texto \cs{bfseries} \{\grupo{texto \cs{itshape} \escopo{texto}}\} texto.
	\end{itemize}\vfill
	
	\begin{block}<4->{Exercício 2\hfill \hyperlink{respostas}{\footnotesize\textbf{(resposta)}}}
		Na atividade anterior, modifique a posição de \emph{uma} das chaves de modo que
		apenas os dois ``texto'' centrais fiquem em negrito.
	\end{block}\vfill
	
	\begin{uncoverenv}<2->
		\footnotesize
		Legenda: \grupo{grupo}\quad\escopo{escopo}
	\end{uncoverenv}
	
\end{frame}
%-------------------------------------------------------------------
\subsection{Escopo de ambientes}
\begin{frame}[fragile]
	\frametitle{O escopo de um ambiente}
	\framesubtitle{é o grupo \emph{automaticamente} definido entre {\sffamily\textbackslash begin} e {\sffamily\textbackslash end}}
	
	\vfill
	
	\begin{columns}
		\column[c]{0.4\textwidth}
		\begin{atividade}{4}%
			\begin{LaTeXcode}%
				texto
				\begin{bfseries}%
				  texto
				  \begin{itshape}%
				    texto
				  \end{itshape}%
				  texto.
				\end{bfseries}
			\end{LaTeXcode}
		\end{atividade}
		
		\column[c]{0.5\textwidth}			
		\onslide<2->{Para que servem os \%?}
		
	\end{columns}
	
	\begin{block}<3->{Exercício 3\hfill \hyperlink{respostas}{\footnotesize\textbf{(resposta)}}}
		Na atividade anterior, modifique a posição do início ou do fim do ambiente \texttt{bfseries}
		de modo que apenas os dois ``texto'' centrais fiquem em negrito.
	\end{block}	
		
\end{frame}
%-------------------------------------------------------------------
\subsection{Revisão: argumentos compulsórios}
\begin{frame}[fragile]
	\frametitle{Revisão}
	\framesubtitle{Argumentos de comandos}
	
		\begin{block}{Exercício 4\hfill \hyperlink{respostas}{\footnotesize\textbf{(resposta)}}}	
			\begin{enumerate}
			\item<1-7> Identifique o \emph{único} argumento de \cs{comando}:
	
				\begin{itemize}\bfseries
					\item[]<2-3> \{A B\} C \cs{comando} \{{\color<3>{red}D E F G}\} H I
					\item[]<4-5> A B C \cs{comando} {\color<5>{red}D} \{E F G\} H I
					\item[]<6-7> A B C \cs{comando} {\color<7>{red}D} {\sffamily\textbackslash LaTeX} \{\} F G H I
				\end{itemize}
			
			\item<8-16> Identifique os \emph{dois} argumentos de \cs{comando}:
	
				\begin{itemize}\bfseries
					\item[]<8-10> \{A B\} C \cs{comando} \{{\color<9>{red}D E F G}\} {\color<10>{red}H} I
					\item[]<11-13> A B C \cs{comando} {\color<12>{red}D} \{{\color<13>{red}E F G}\} H I
					\item[]<14-16> A B C \cs{comando} {\color<15>{red}D} {\color<16>{red}\sffamily\textbackslash LaTeX} \{\} F G H I
				\end{itemize}
	
			\item<17-28> Identifique os \emph{três} argumentos de \cs{comando}:
	
				\begin{itemize}\bfseries
					\item[]<17-20> \{A B\} C \cs{comando} \{{\color<18>{red}D E F G}\} {\color<19>{red}H} {\color<20>{red}I}
					\item[]<21-24> A B C \cs{comando} {\color<22>{red}D} \{{\color<23>{red}E F G}\} {\color<24>{red}H} I
					\item[]<25-28> A B C \cs{comando} {\color<26>{red}D} {\color<27>{red}\sffamily\textbackslash LaTeX} {\color<28>{red}\{\}} F G H I
				\end{itemize}
			\end{enumerate}
		\end{block}
		
		\vfill
		
		\begin{center}
			\bfseries
			\textbf{Lembre-se:} os espaços entre os argumentos são removidos!
		\end{center}
		
\end{frame}
%-------------------------------------------------------------------
\section{Instruções mutuamente exclusivas}
\begin{frame}[fragile]
	\frametitle{Instruções mutuamente exclusivas}
		
	\begin{atividade}{5}
		\begin{LaTeXcode}
			{texto \slshape\sffamily texto}
			{texto \slshape\scshape texto}
		\end{LaTeXcode}
	\end{atividade}

	\begin{atividade}<2->{6}
		\begin{LaTeXcode}
			texto \textsl{\textsf{texto}}
			texto \textsl{\textsc{texto}}
		\end{LaTeXcode}
		
		\footnotesize\textbf{obs.:} note que a interpretação das instruções pelo \LaTeX\ ocorre
		da esquerda para a direita, isto é, vale a última instrução dada.
	\end{atividade}

	\begin{block}<3->{Exercício 5\hfill \hyperlink{respostas}{\footnotesize\textbf{(resposta)}}}
		Recrie os exemplos acima usando apenas ambientes.
	\end{block}
	
\end{frame}
%-------------------------------------------------------------------
\section{Intersecção de grupos}
\begin{frame}[fragile]
	\frametitle{Intersecção de grupos}
	
	\begin{itemize}
	\item \textbf{Atenção:} diferentemente da intersecção de ambientes
	definidos por chaves, que é interpretada ``incorretamente,'' a intersecção
	de ambientes gera erro! Por exemplo, o código abaixo não compila:
	\end{itemize}
	
	\begin{atividade}{7}
		\begin{LaTeXcode}		
			\begin{slshape}
			  \begin{sffamily}
			    Há um erro aqui!
			\end{slshape}
			  \end{sffamily}
		\end{LaTeXcode}
	\end{atividade}
	
\end{frame}
%-------------------------------------------------------------------
\section{Alinhamento do texto}

\begin{frame}[fragile]
	\frametitle{Alinhamento do texto}
	
	\begin{atividade}<1->{8}
		\begin{LaTeXcode}
			\centering Texto\\centralizado.
			\raggedleft Texto alinhado\\à direita.
			\raggedright Texto alinhado\\à esquerda.
		\end{LaTeXcode}
	\end{atividade}\vfill
				
	\begin{atividade}<2->{9}
		Utilize o pacote \pkg{lipsum} (\foreign{lorem ipsum}) neste arquivo:
		\begin{LaTeXcode}		
			\begin{center}\lipsum[1]\end{center}
			\begin{flushright}\lipsum[2]\end{flushright}		
			\begin{flushleft}\lipsum[3]\end{flushleft}
		\end{LaTeXcode}
	\end{atividade}\vfill		
				
	\centering
	\begin{uncoverenv}<1->
		\footnotesize\bfseries
		Atenção: evite utilizar \verb|\\| para controlar a distância entre as linhas e
		\\\emph{jamais} utilize-o para controlar a distância entre parágrafos!
	\end{uncoverenv}
			
\end{frame}
%-------------------------------------------------------------------
{
	\logo{\includegraphics[width=0.25\textwidth]{LogotipoCursoLaTeX_v3_pequeno}}
	\setbeamertemplate{background canvas}{}
	\againframe{titulo} % Reapresenta a página inicial.
}
%-------------------------------------------------------------------
%-------------------------------------------------------------------
%-------------------------------------------------------------------
\section{Respostas}
\begin{frame}[fragile,label=respostas]
	\frametitle{Respostas}
	\scriptsize
	
	\begin{enumerate}
	\item%
		\begin{LaTeXcode}
			texto \textbf{texto \textit{texto}} texto.
		\end{LaTeXcode}
		
	\item%
		\begin{LaTeXcode}
			texto {\bfseries texto \itshape texto} texto.
		\end{LaTeXcode}
		
	\item%
		\begin{LaTeXcode}
			texto \begin{bfseries}texto\begin{itshape}
			texto\end{itshape}\end{bfseries} texto
		\end{LaTeXcode}
		
	\item Resposta no próprio enunciado
	
	\item%
		\begin{LaTeXcode}
		{texto\begin{slshape}\begin{sffamily}
		texto\end{sffamily}\end{slshape}}
		\end{LaTeXcode}
		
		\begin{LaTeXcode}
		{texto\begin{slshape}\begin{scshape}
		texto\end{scshape}\end{slshape}}
		\end{LaTeXcode}
	\end{enumerate}
	
	\textbf{obs.:} nos exercícios com ambientes, procure identificar a origem dos espaços entre as palavras
	
\end{frame}
\end{document}