\documentclass[a4paper,12pt]{article}
	\usepackage[ansinew]{inputenc}
	\usepackage{ae}	
	\usepackage[brazil]{babel}
\begin{document}

	Você pode colocar uma \fbox{moldura} ao redor de um pedaço de texto, ao
	redor de uma expressão matemática, como \fbox{$y(x) = \sin(x)$}, ou mesmo
	ao redor de uma figura.\footnote{Mas para isso você vai ter de esperar
	um pouco mais}
				
	\begin{center}
		\fbox{$  \sec^2\theta = 1 + \tan^2\theta  $}
	\end{center}
	
	% obs.: você NÃO pode utilizar \[ ou $$ (estilo de exibição) acima: dentro de
	% uma caixa LR é proibido inserir uma expressão que tenha sua própria linha.
	% Para fazer isso, você deve colocá-la dentro de uma caixa de parágrafo e só
	% então utilizar \fbox ou \framebox:
	%
	% \fbox{\parbox{0.5\textwidth}{$$  \sec^2\theta = 1 + \tan^2\theta  $$}}
	
\end{document}