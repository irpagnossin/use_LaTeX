\documentclass[a4paper,12pt]{article}
	\usepackage[utf8]{inputenc}
\begin{document}

	\mdseries Série média
	
	\bfseries Série negrito expandido (ou boldface)
	
	% Compare a palavra "série" nas duas frases acima e responda: por que o
	% nome da série é "negrito EXPANDIDO" e não apenas "negrito".
	
\end{document}

% obs.: neste arquivo, como utilizamos o caracteres acentuados, ASCII não serve
% (resultaria em erro ao compilar). Então, precisamos utilizar a opção "cp1252"
% ou "ansinew" (são nomes diferentes para a mesma codificação), que é a codificação
% padrão utilizada pelo Windows. Em geral também é possível utilizarmos a opção
% "utf8" (codificação Latin-1 ou ISO-8859-1), pois os caracteres que costumamos
% digitar num documento de instrução como este estão presentes tanto na Latin-1
% como na CP-1252.