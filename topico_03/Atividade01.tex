\documentclass[a4paper,12pt]{article}
  \usepackage[utf8]{inputenc} % <-- ATENÇÃO: \usepackage só no preâmbulo!
\begin{document}
	Agora é possível acentuar o texto normalmente,
	mas você ainda pode usar a acentua\c c\~ao
	padr\~ao do \TeX, se quiser!
\end{document}

% A partir de agora você pode - e deve - utilizar o pacote inputenc, não apenas
% porque ele lhe permite acentuar normalmente os caracteres, mas principalmente
% porque assim você especifica explicitamente qual é a codificação que o LaTeX
% deve utilizar para compilar com sucesso o seu arquivo.

% Mesmo quando você estiver escrevendo um documento em inglês, onde caracteres
% acentuados raramente são requeridos (e, quando são, podem ser obtidos conforme
% a acentuação padrão do LaTeX), é bom utilizar o pacote inputenc com a opção
% "ascii" (isto é, no lugar do "utf8") para, novamente, deixar claro qual é a
% codificação que você utilizou para escrever seu arquivo de instruções.