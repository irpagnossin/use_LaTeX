\documentclass{article}

	\usepackage[ansinew]{inputenc} % <-- Este arquivo foi escrito com a codificação CP-1252.
	\usepackage[T1]{fontenc}       % <-- Instrui o LaTeX a utilizar fontes de 8 bits.
	\usepackage{palatino}          % <-- Substitui a fonte Computer Modern Roman, Sans-serif e Typewriter por Palatino, Helvetica e Courier NO TEXTO
	%\usepackage{mathpazo}          % <-- Substitui a fonte Computer Modern Roman por Palatino NO TEXTO e por Pazo NO MODO MATEMÁTICO.
	\usepackage[brazil]{babel}     % <-- Carrega os padrões de separação silábica do português.
	
\begin{document}

	A qualidade na integração entre texto e expressões matemáticas, característica do \LaTeX, reside na configuração harmônica entre as famílias de fontes utilizadas no texto e nas expressões. Por exemplo, neste arquivo nós utilizamos o pacote ``palatino'' para substituir a Computer Modern Roman pela família Palatino \emph{no texto} (além disso, este pacote substitui também as Computer Moder Sans-Serif e Typewriter por Helvetica e Courier, respectivamente). Mas no modo matemático, como em $a^2 = b^2 + c^2$, a fonte continua sendo a Computer Modern Roman. Outro exemplo:		
	\[ % <-- Entra no modo matemático.
	\frac{d}{dt}\int_0^t \lambda^2\, d\lambda.
	\] % <-- Sai do modo matemático.

	Assim, utilizar o pacote ``palatino'' destrói a harmonia entre as fontes do modo texto e do modo matemático.
	
	No caso da família Palatino, para corrigir isto basta substituir o pacote ``palatino'' pelo ``mathpazo,'' que encarrega-se de definir também a fonte do modo matemático como Pazo, que combina com a Palatino (neste caso, a substituição das fontes Computer Modern Sans-Serif e Typewriter não ocorre). Faça o teste!
	
	A família de fontes Palatino pertence ao estilo chamado Romano ou antigo. Famílias deste estilo têm serifas curvilíneas e traçados que apresentam a chamada transição grosso-fino, isto é, transições contínuas e suaves entre grosso e fino, como se tivessem sido desenhados à mão. Outros exemplos de famílias deste \emph{estilo} são a própria Computer Modern Roman, a Bookman, a Times, dentre muitas outras. Este é o estilo de fontes mais recomendado para a impressão de longos trabalhos.
	
	Contudo, note que o \emph{estilo} de uma família de fontes não faz parte da classificação dela segundo o New Font Selection Scheme (NFSS).

\end{document}