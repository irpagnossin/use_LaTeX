%-------------------------------------------------------------------------------
% Autores: I. R. Pagnossin e Centro de Ensino e Pesquisa Aplicada.
%
% Este material é parte integrante do curso "Usando LaTeX; pensando em TeX" e é
% distribuido pelos autores segundo a licença Creative Commons 2.5 Brasil
% (atribuição/não-comercial/redistribuição segundo a mesma licença).
%
% This material is part of the course "Usando LaTeX; pensando em TeX".
% It is distributed according to the license Creative Commons 2.5 Brazil
% (atribution/non-comercial use/share alike the same license).
%-------------------------------------------------------------------------------
\newif\ifhandout
%\handouttrue  % Descomente se for para gerar a versão para IMPRESSÃO.
\handoutfalse % Descomente se for para gerar a versão para APRESENTAÇÃO

%-------------------------------------------------------------------------------
\ifhandout
  \documentclass[handout,10pt]{beamer}
  \mode<handout>
\else
	\documentclass[10pt]{beamer}
	\mode<presentation>
\fi

	\usepackage[utf8]{inputenc}	
	\usepackage[brazil]{babel}	
	\usepackage{graphicx}
	\usepackage{listings}
	\usepackage{tikz}
	\usepackage{helvet}	
	\usepackage{amsmath}
	\usepackage{bookman}
	\usepackage{helvet}		
	\usepackage[squaren]{SIunits}
	\usepackage{booktabs}
	
	\ifhandout
		\usepackage{pgfpages}
		\pgfpagesuselayout{2 on 1}[a4paper,border shrink=5mm]
	\fi
	
	

	\usetikzlibrary{shapes.symbols}


	% Configurações pessoais
	% Configurações personalizadas do código LaTeX.	
\lstnewenvironment{LaTeXcode}{
	\setlength{\abovecaptionskip}{0pt}	
	\lstset{language=[LaTeX]TeX}
	\lstset{%
		basicstyle=\footnotesize\ttfamily,  % Global
		keywordstyle=\color{blue}\bfseries, % Comandos
		identifierstyle=,                   % Texto
		stringstyle=,                       % Strings 
		commentstyle=\color{gray},          % Comentários
		showstringspaces=false,             % Espaços
		rulecolor=\color{gray},             % Linha da caixa
	}
	\lstset{emph={setlength,includegraphics,psfrag,subfigure},emphstyle={\color{blue}\bfseries}}
}% Abrindo o ambiente.
{}% Fechando o ambiente.
	
\newcommand{\digite}[1]{{\fontfamily{cmss}\fontseries{bx}\selectfont#1}}	
\newcommand{\cs}[1]{{\normalfont\textbackslash\color{blue!50!black}#1}}
\newcommand{\pkg}[1]{{\normalfont\sffamily\color{orange}#1}}
\newcommand{\env}[1]{{\normalfont\sffamily\color{green!50!black}#1}}
\let\comando=\cs
\let\package=\pkg
\let\ambiente=\env
\newcommand{\foreign}[1]{{\textsl{#1}}}


	\newcounter{exercicio}	
	\newenvironment{exercicio}{%
		\refstepcounter{exercicio}%
		\penalty-200
		\noindent\colorbox{blue!60!black}{\makebox[\columnwidth-\fboxsep*2][c]{\textbf{\color{white}Exercício~\theexercicio}}}\smallskip
	}{\par\medskip}
		

\newcommand{\bibtex}{\textsc{Bib}\TeX}

\newenvironment<>{atividade}[1]{%
\begin{actionenv}#2%
\begin{exampleblock}{{Atividade #1}}%
}
{%
\end{exampleblock}%
\end{actionenv}%
}


	% Path das figuras, relativo a esta pasta.
	\graphicspath{{../arquivos_comuns/figuras/}{./figuras/}}

	% Modelo da apresentação	
	\usetheme{Frankfurt}
	\usefonttheme{serif,structurebold}
	\setbeamercovered{transparent}
	
	% Metadados do arquivo PDF.
	\hypersetup{
		pdftitle={Os primeiros passos},
		pdfauthor={Dr. Ivan R. Pagnossin},
		pdfsubject={LaTeX},
		pdfkeywords={TeX,LaTeX}
	}

	% Título, autores e instituição.
	\title{Os primeiros passos}
	\subtitle{regras básicas, caracteres especiais e comandos}
	
	\author{\textbf{Prof.:} Ivan R. Pagnossin \and \textbf{Tutora:} Juliana Giordano}
	\institute{%
		Coordenadoria de Tecnologia da Informação\\
		Centro de Ensino e Pesquisa Aplicada}
	\logo{\includegraphics[width=0.25\textwidth]{LogotipoCursoLaTeX_v3_pequeno}}
	\date{}
	
	

\begin{document}
%-------------------------------------------------------------------
\begin{frame}[label=titulo]
	\centering	
	
	\includegraphics[width=0.8\textwidth]{LogotipoCursoLaTeX_v2}

	\titlepage
\end{frame}

\logo{} % <-- O logotipo não aparecerá mais a partir daqui.

\setbeamertemplate{background canvas}{%
		\includegraphics[width=\paperwidth,height=\paperheight,keepaspectratio=false]{leao-pensador-wattermark.png}
}
		
%-------------------------------------------------------------------
\section{Introdução ao TeXnicCenter}
\begin{frame}[fragile]
	\frametitle{Introdução ao TeXnicCenter}
	
	\begin{exampleblock}<1->{Atividade 1}
		Configure os \emph{perfís de compilação} do TeXnicCenter e explore um pouco as suas janelas,
		botões e menus.
	\end{exampleblock}
	
	\begin{exampleblock}<2->{Atividade 2}
		Crie o arquivo abaixo, salve-o, compile-o para \textsf{dvi} e veja-o com o Yap, \emph{tudo através do TeXnicCenter}:
		
		\begin{LaTeXcode}
			\documentclass{article}
			\begin{document}
			Meu segundo arquivo \LaTeX!
			\end{document}		
		\end{LaTeXcode}
	\end{exampleblock}
	
	\begin{exampleblock}<3->{Atividade 3}
		Altere o perfil de compilação, recompile e veja o \emph{arquivo de saída}
		(usando o TeXnicCenter).
	\end{exampleblock}	
\end{frame}
%-------------------------------------------------------------------
\section{As 5 regras básicas}

\subsection{1ª regra}
\begin{frame}[fragile,label=regra1]
	\frametitle{As 5 regras básicas}
	\framesubtitle{1ª regra}
	
	\begin{center}
		\begin{uncoverenv}<1->		
			\small Elas referem-se a como o \LaTeX\ ``enxerga'' o arquivo de instruções e\\
			permitem ainda que você possa organizá-lo de forma inteligível.
		\end{uncoverenv}
	\end{center}
	
	\begin{block}<2->{1ª regra}
		Um espaço é tão bom quanto dez
	\end{block}
	
	\begin{exampleblock}<3->{Atividade 4}
		\begin{LaTeXcode}
			\documentclass{article}
			\begin{document}
			Meu segundo  arquivo          \LaTeX!
			\end{document}
		\end{LaTeXcode}
	\end{exampleblock}	
\end{frame}
%-------------------------------------------------------------------
\subsection{2ª regra}
\begin{frame}[fragile,label=regra2]
	\frametitle{As 5 regras básicas}
	\framesubtitle{2ª regra}
	
	\begin{block}<1->{2ª regra}
		\alert{Uma} quebra de linha equivale a um espaço
	\end{block}
	
	\begin{exampleblock}<2->{Atividade 5}
		\begin{LaTeXcode}
			\documentclass{article}
			\begin{document}
			Meu
			segundo
			arquivo
			\LaTeX!
			\end{document}
		\end{LaTeXcode}
	\end{exampleblock}
	
	\begin{exampleblock}<3->{Atividade 6}
		Reescreva o arquivo acima numa única linha
	\end{exampleblock}
\end{frame}
%-------------------------------------------------------------------
\subsection{3ª regra}
\begin{frame}[fragile,label=regra3]
	\frametitle{As 5 regras básicas}
	\framesubtitle{3ª regra}
	
	\begin{block}<1->{3ª regra}
		Uma linha em branco indica mudança de parágrafo
	\end{block}
	
	\begin{exampleblock}<2->{Atividade 7}
		\begin{LaTeXcode}
			\documentclass{article}
			\begin{document}
			texto texto texto...
			
			texto texto texto...
			\end{document}
		\end{LaTeXcode}
	\end{exampleblock}	
\end{frame}
%-------------------------------------------------------------------
\subsection{4ª regra}
\begin{frame}[fragile,label=regra4]
	\frametitle{As 5 regras básicas}
	\framesubtitle{4ª regra}
	
	\begin{block}<1->{4ª regra}
		Uma linha em branco é tão boa quanto dez
	\end{block}
	
	\begin{exampleblock}<2->{Atividade 8}
		\begin{LaTeXcode}
			\documentclass{article}
			\begin{document}
			texto texto texto...
			
			
			
			texto texto texto...
			\end{document}
		\end{LaTeXcode}
	\end{exampleblock}	
\end{frame}
%-------------------------------------------------------------------
\subsection{5ª regra}
\begin{frame}[fragile,label=regra5]
	\frametitle{As 5 regras básicas}
	\framesubtitle{5ª regra}
	
	\begin{block}<1->{5ª regra}
		Espaços no início de linha são ignorados
	\end{block}
	
	\begin{exampleblock}<2->{Atividade 9}
		\begin{LaTeXcode}
			\documentclass{article}
			\begin{document}
			      Meu segundo arquivo \LaTeX!
			\end{document}
		\end{LaTeXcode}
	\end{exampleblock}	
\end{frame}
%-------------------------------------------------------------------
\subsection{Resumo}
\begin{frame}
	\frametitle{As 5 regras básicas}
	\framesubtitle{Resumo}
	
	\makebox[\textwidth]{%
	\begin{minipage}{1.15\textwidth}
	\begin{description}
		\item[1ª regra:] \hyperlink{regra1}{Um espaço é tão bom quanto dez}
		\item[2ª regra:] \hyperlink{regra2}{\emph{Uma} quebra de linha equivale a um espaço}
		\item[3ª regra:] \hyperlink{regra3}{Uma linha em branco indica mudança de parágrafo}
		\item[4ª regra:] \hyperlink{regra4}{Uma linha em branco é tão boa quanto dez}
		\item[5ª regra:] \hyperlink{regra5}{Espaços no início de linha são ignorados}
	\end{description}
	\end{minipage}}
\end{frame}
%-------------------------------------------------------------------
%-------------------------------------------------------------------
\section{Caracteres especiais}
\begin{frame}[fragile]
	\frametitle{Caracteres especiais}
	
	\centering
	
	{\small No \LaTeX, alguns caracteres têm significado especial e, por isso,\\
	\textbf{não podem ser utilizados normalmente no texto}. São eles:\par}
	
	\begin{block}{}
		\centering
		\begin{tabular}{lll}
			Caráter							& Significado									& Para imprimí-lo use \\
			\midrule
			\verb|\|            & \alert<1>{Inicia comandos}  & \verb|\textbackslash| \\
			\verb|{| e \verb|}| & \alert<1>{Agrupamento}			& \verb|\{| e \verb|\}| \\
			\verb|%|            & Comentários                 & \verb|\%| \\
			\verb|~|            & Espaço indivisível          & \verb|\~{}| \\
			\verb|$|            & Modo matemático             & \verb|\$| \\
			\verb|_|            & Subscrito/índice            & \verb|\_| \\
			\verb|^|            & Sobrescrito/expoente        & \verb|\^{}| \\
			\verb|&|            & Tabulação                   & \verb|\&| \\
			\verb|#|            & Ordenação de parâmetros     & \verb|\#| \\
		\end{tabular}
	\end{block}
		
	\begin{block}{Exercício 1\hfill \hyperlink{R1a3}{\footnotesize\textbf{(resposta)}}}
		Gere um documento com todos os caracteres especiais acima
	\end{block}	
\end{frame}
%-------------------------------------------------------------------
\begin{frame}[fragile]
	\frametitle{Comentários}
	
	\centering
	
	{\small Comentários são informações úteis ao autor,	mas não pertinentes	ao documento final.
	Por isso, eles \emph{não} aparecem nele.\par}
	
	\vfill
	
	\begin{exampleblock}<2->{Atividade 10}
		\begin{LaTeXcode}
			\documentclass{article}
			\begin{document}
			Meu segundo % Isto é um comentário.
			     arquivo \LaTeX!
			\end{document}		
		\end{LaTeXcode}
		
		\uncover<4->{Altere a posição do caráter \% e veja o que acontece.}
	\end{exampleblock}
	
	\vfill
	
	\onslide<3->{Tudo o que estiver entre \% e o final da linha é
	ignorado pelo \LaTeX, isto é, não aparece no documento final.}
	
\end{frame}
%-------------------------------------------------------------------
\begin{frame}[fragile]
	\frametitle{Exercícios}	

	\begin{block}<2->{Exercício 2\hfill \hyperlink{R1a3}{\footnotesize\textbf{(resposta)}}}
		No arquivo da atividade anterior, coloque \% imediatamente após o último caráter da primeira linha, identifique a mudança no arquivo de saída e explique-a.
	
		\hrulefill\par%
		\footnotesize\textbf{Dica:} regras básicas 2 e 4.		
	\end{block}

	\begin{block}<3->{Exercício 3\hfill \hyperlink{R1a3}{\footnotesize\textbf{(resposta)}}}
		Por que \alert{não} há dois parágrafos aqui?
		
		\begin{LaTeXcode}
			texto texto texto... %
			                     %
			texto texto texto...
		\end{LaTeXcode}
		
		\hrulefill\par%
		\footnotesize\textbf{Dica:} regra básica 5.			
	\end{block}
\end{frame}
%-------------------------------------------------------------------
%-------------------------------------------------------------------
\section{Comandos}

\subsection{Tipos}
\begin{frame}[fragile]
	\frametitle{Comandos ou \foreign{control sequences}}
	\framesubtitle{Tipos}
		
	\begin{block}<2->{1º tipo \hfill \foreign{control words}}
		Barra-invertida seguida de uma sequência arbitrariamente longa de caracteres exclusivamente alfabéticos.
	\end{block}

	\begin{uncoverenv}<3->	
		\begin{tabbing}
			Exemplos: \= \verb|\documentclass|\+\\
									 \verb|\begin|          \\
									 \verb|\end|            \\
									 \verb|\LaTeX|          \\
								   etc
		\end{tabbing}
	\end{uncoverenv}

	\begin{block}<4->{2º tipo \hfill \foreign{control symbols}}
		Barra-invertida seguida de \alert{um} caráter não-alfabético.
	\end{block}
	
	\begin{uncoverenv}<5->
		\begin{tabbing}
			Exemplos: \= \verb|\_|\+\\
								   \verb|\~|  \\
								   \verb|\#|  \\
								   etc
		\end{tabbing}
	\end{uncoverenv}		
\end{frame}
%-------------------------------------------------------------------
\begin{frame}[fragile]
	\frametitle{Comandos ou \foreign{control sequences}}
	\framesubtitle{Exercícios}
	
	\centering
	
	\begin{block}{Exercício 4\hfill \hyperlink{R4a6}{\footnotesize\textbf{(resposta)}}}
		Identifique o tipo dos comandos abaixo:
		\begin{enumerate}
		\item<1,8-> \verb|\textbackslash|		
		\item<2,8-> \verb|\+|
		\item<3,8-> \verb|\\|
		\item<4,8-> \verb|\EsteComandoEMuitoLongoNaoEMesmo|
		\item<5,8-> \verb|\a|
		\item<6,8-> \verb|\-|		
		\item<7,8-> \verb|\ |{\footnotesize\qquad$\leftarrow$ tem um espaço após a barra}
		\end{enumerate}
	\end{block}
	
	\vfill
	
	\uncover<8->{\textbf{Atenção:} o \LaTeX\ diferencia maiúsculas de minúsculas.}
		
\end{frame}
%-------------------------------------------------------------------
\subsection{Parâmetros}

\begin{frame}[fragile,shrink=0.1]
	\frametitle{Comandos ou \foreign{control sequences}}
	\framesubtitle{Argumentos}

	\centering

	\begin{block}{Comandos que não recebem argumentos}
		\cs{comando}
	\end{block}
	
	\begin{tabbing}
		Exemplos de comandos s/ argumento: \=\verb|\LaTeX|\`\LaTeX\+\\
						                             \verb|\%|\`\%\\
						                             \verb|\textbackslash|\`\textbackslash
	\end{tabbing}
		
	\begin{block}<2->{Comandos que recebem argumentos}	
		\cs{comando} \{\textit{argumento 1}\} \{\textit{argumento 2}\}\dots
	\end{block}
	
	\begin{uncoverenv}<2->
		\begin{tabbing}
			Exemplos de comandos c/ argumento: \=\verb|\begin{document}| \+\\
			                                     \verb|\^{g}|\`\^g\\
			                                     \verb|\~{g}|\`\~g
		\end{tabbing}
	\end{uncoverenv}
		
\end{frame}
%-------------------------------------------------------------------
\setlength{\fboxsep}{0pt}
\newcommand{\argA}[1]{{\colorbox{red!20}{\bfseries\{#1\}}}}
\newcommand{\argB}[1]{{\colorbox{green!20}{\bfseries\{#1\}}}}

\begin{frame}[fragile,t]
	\frametitle{Comandos ou \foreign{control sequences}}
	\framesubtitle{Argumentos --- regra A}
	
	\begin{block}{Regra A}
		Um comando toma os \alert{caracteres} ou \alert{grupos} que o segue como argumentos
	\end{block}

	\onslide<2->{Imagine que \cs{comando} requeira \alert{um} argumento\dots}

	\begin{itemize}\bfseries
		\item[]< 2-> \cs{comando} {\colorbox{red!20}A} B C
		\item[]< 4-> \cs{comando} \argA{A} B C
		\item[]< 6-> \cs{comando} \argA{A B} C\hfill eg.: \verb|\begin{document}|
	\end{itemize}	

	\onslide<3->{Imagine que \cs{comando} requeira \alert{dois} argumentos\dots}

	\begin{itemize}\bfseries
		\item[]< 3-> \cs{comando} {\colorbox{red!20}A} {\colorbox{green!20}B} C
		\item[]< 5-> \cs{comando} \argA{A} \argB{B} C
		\item[]< 7-> \cs{comando} \argA{A B C} \argB{?}\hfill \alert<7->{\mathversion{bold}$\leftarrow$ ERRO!}
	\end{itemize}
\end{frame}
%-------------------------------------------------------------------
\begin{frame}[fragile,t]
	\frametitle{Comandos ou \foreign{control sequences}}
	\framesubtitle{Argumentos --- regra B}
	
	\centering
	
	\begin{block}<1->{Regra B}
		Todos os espaços existentes entre o comando e seus argumentos e entre os argumentos são removidos
	\end{block}
			
	\begin{exampleblock}<2->{Atividade 11}		
		\begin{LaTeXcode}
			\comando a    \LaTeX   {texto qualquer}  FG
		\end{LaTeXcode}
		
		\textbf{obs.:} use o arquivo \verb|Atividade11.tex| para esta atividade.
	\end{exampleblock}
	
	\begin{block}<3->{Exercício 5\hfill \hyperlink{R4a6}{\footnotesize\textbf{(resposta)}}}
		Na atividade 11, remova o espaço entre \verb|\comando| e a letra ``a'' e	recompile. Por que ocorre erro de compilação?
	\end{block}
\end{frame}
%-------------------------------------------------------------------
\begin{frame}[fragile]
	\frametitle{Comandos ou \foreign{control sequences}}
	\framesubtitle{Exemplos}

	Exemplos:
	\begin{itemize}
	\item<+-> \verb|\documentclass{article}|
	\item<+-> \verb|\documentclass      {article}|
	\item<+-> \verb|\documentclass article|
	\item<+-> \verb|\documentclass    article|
	\item<+-> \verb|\~{a}|
	\item<+-> \verb|\~a|
	\item<+-> \verb|\~   a|
	\end{itemize}

	\begin{block}<8->{Exercício 6\hfill \hyperlink{R4a6}{\footnotesize\textbf{(resposta)}}}
		A instrução abaixo resulta em erro de compilação. Por quê?
		
		\begin{LaTeXcode}
		\documentclass article
		\end{LaTeXcode}
	\end{block}
\end{frame}
%-------------------------------------------------------------------
\begin{frame}[fragile]
	\frametitle{Comandos ou \foreign{control sequences}}
	\framesubtitle{Argumentos}
	
	\centering
	
	{\small	Por consequência da regra B, mesmo no caso dos comandos que não recebem argumentos, os espaços após ele são removidos\par}
				
	\begin{exampleblock}{Atividade 12}
		\begin{LaTeXcode}
				texto \LaTeX   texto
				 
				texto \LaTeX\  texto 
				
				texto \LaTeX{} texto
		\end{LaTeXcode}
	\end{exampleblock}	
		
	\vfill
	
	\begin{uncoverenv}<2->
		\small Para evitar	que isto aconteça, utilize \verb*|\ | ou \verb|{}| após o comando.\par
		{\footnotesize\textbf{obs.:} neste caso as chaves não indicam argumento, mas um grupo vazio.\par}
	\end{uncoverenv}
	
	\vfill
	
	\begin{block}<3->{Exercício 7\hfill \hyperlink{R7a9}{\footnotesize\textbf{(resposta)}}}
		Eu estou aprendendo \LaTeX\ na USP.
	\end{block}
	
\end{frame}
%-------------------------------------------------------------------
\begin{frame}[fragile]
	\frametitle{Comandos ou \foreign{control sequences}}
	\framesubtitle{Argumentos opcionais}

	\centering

	\begin{block}{}
		\cs{comando}[\textit{arg. opcional}]\{\textit{arg. compulsório 1}\}\dots		
	\end{block}
	
	{\small	Alguns comandos podem receber argumentos opcionais,	que são \alert{obrigatoriamente} passados entre colchetes. A posição deles depende do comando, mas como regra geral eles antecedem os argumentos compulsórios.\par}
		
	\begin{exampleblock}<2->{Atividade 13}
		No arquivo anterior, altere a primeira linha do arquivo para
		
		\begin{LaTeXcode}
			\documentclass[a4paper,12pt]{article}
		\end{LaTeXcode}
	\end{exampleblock}
	
	\small
	\begin{description}
		\item[a4paper]<3-> Define o formato do papel como A4 (210 $\times$ \unit{297}{\milli\metre}).
		\item[12pt]<4-> Define o tamanho padrão da fonte: \unit{12}{pt}.
	\end{description}
	
\end{frame}
%-------------------------------------------------------------------
\subsection{Declarações e ambientes}
\begin{frame}
	\frametitle{Comandos ou \foreign{control sequences}}
	\framesubtitle{Declarações e ambientes}
	
	Os comandos do \alert{primeiro tipo} (\foreign{control words}) podem ser:
	
	\vfill
	
	\begin{enumerate}
		\item<2-> \textbf{Declarações} (\foreign{declarations}): {\footnotesize comandos que não produzem nada
			no documento final, mas afetam as instruções seguintes.\par}
			
			{\footnotesize Exemplo: \cs{documentclass}\{\textit{article}\}.}
	
			\begin{block}<3->{Exercício 8\hfill \hyperlink{R7a9}{\footnotesize\textbf{(resposta)}}}
				Por que \cs{documentclass} é uma declaração?
			\end{block}
	
			\vfill
			
		\item<4-> \textbf{Ambientes} (\foreign{environments}): {\footnotesize são aqueles comandos iniciados
			por \cs{begin}\{\textit{nome}\} e terminados por \cs{end}\{\textit{nome}\}. \alert<5>{\textit{nome} é o
			nome deste ambiente}. \alert<6>{Sempre deve haver um {\sffamily\textbackslash begin} para cada 
			{\sffamily\textbackslash end}, e vice-versa}.\par}
		
			\vfill
		
			\begin{block}<7->{Exercício 9\hfill \hyperlink{R7a9}{\footnotesize\textbf{(resposta)}}}
				Que exemplo de ambiente você conhece?
			\end{block}				
	\end{enumerate}	
\end{frame}
%-------------------------------------------------------------------
{
	\logo{\includegraphics[width=0.25\textwidth]{LogotipoCursoLaTeX_v3_pequeno}}
	\setbeamertemplate{background canvas}{}
	\againframe{titulo} % Reapresenta a página inicial.
}
%-------------------------------------------------------------------
%-------------------------------------------------------------------
%-------------------------------------------------------------------
\section{Respostas dos exercícios}

\begin{frame}[fragile,label=R1a3]
	\frametitle{Respostas}
	
	\begin{enumerate}\footnotesize
	\item Coloque estas instruções no ambiente \foreign{document}:
		\begin{LaTeXcode}
		\textbackslash, \{ e \}, \~{}, \$, \_, \^{}, \&, \# e \%
		\end{LaTeXcode}
	
	\item \textbf{Mudança:} some o espaço entre ``segundo'' e ``arquivo\rlap.'' \textbf{Motivo:} o \% comenta o caráter de quebra de linha, impedindo o \LaTeX\ de atribuir um espaço a ele (regra 2). Além disso, os espaços no início da segunda linha são ignorados (regra 4). Como resultado, não existe espaço entre o	último caráter da primeira linha e o primeiro da segunda.
	
	\item A linha ``vazia'' que supostamente indicaria a mudança de parágrafo é totalmente ignorada pelo \LaTeX. Por dois motivos: primeiro porque como o \% é o primeiro caráter da linha, todos os espaços anteriores a ele são ignorados (regra 5). Segundo porque o \% faz o \LaTeX\ ignorar tudo o que está depois dele, inclusive ele próprio e o caráter de quebra de linha. Assim, do ponto de vista do \LaTeX\ esta linha não existe.
	\end{enumerate}
\end{frame}
%-------------------------------------------------------------------
\begin{frame}[fragile,label=R4a6]
	\frametitle{Respostas}
	
	\begin{enumerate}\footnotesize\setcounter{enumi}{3}
	\item \textbf{Atenção:} a existência ou não de argumentos nada tem a ver com o tipo do comando.
		\begin{enumerate}\footnotesize
		\item \verb|\textbackslash|\hfill 1º tipo	
		\item \verb|\+|\hfill 2º tipo
		\item \verb|\\|\hfill 2º tipo
		\item \verb|\EsteComandoEMuitoLongoNaoEMesmo|\hfill 1º tipo
		\item \verb|\a|\hfill 1º tipo
		\item \verb|\-|\hfill 2º tipo
		\item \verb|\ |{\footnotesize\qquad$\leftarrow$ tem um espaço após a barra}\hfill 2º tipo
		\end{enumerate}
	
	\item Por que o \LaTeX\ não conhece o comando \verb|\comandoa|
	
	\item O argumento do comando \verb|\documentclass| é a letra ``a'' de ``article\rlap.'' Esta construção gera erro ao ser compilada porque o \LaTeX\ não conhece uma classe de documento chamada ``a'' (não existe um arquivo \verb|a.cls| na distribuição). Para corrigir isso precisamos \alert{agrupar} os caracteres da palavra ``article'' usando chaves: \verb|{article}|. \textbf{obs.:} mesmo que existisse tal classe, ainda assim ocorreria erro por que o \emph{texto} ``rticle'' (sem o ``a'') não podem existir no preâmbulo, mas apenas no corpo do documento.	
	
	\end{enumerate}
\end{frame}
%-------------------------------------------------------------------
\begin{frame}[fragile,label=R7a9]
	\frametitle{Respostas}
	
	\begin{enumerate}\footnotesize\setcounter{enumi}{6}	
	
	\item Coloque estas instruções no ambiente \foreign{document}:
		\begin{LaTeXcode}
		Eu estou aprendendo \LaTeX\ na USP.
		\end{LaTeXcode}
		
	\item \verb|\documentclass| é uma declaração porque não imprime nada no documento final, mas afeta tudo o que lhe segue: as dimensões da folha, as margens, os formatos de capítulos e seções etc.
	
	\item O ambiente \foreign{document}: \verb|\begin{document}|\dots\verb|\end{document}|
	\end{enumerate}	
\end{frame}
%-------------------------------------------------------------------

\end{document}
